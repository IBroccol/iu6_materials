\documentclass[11pt]{article}
% for russian lang
\usepackage[T2A]{fontenc}
\usepackage[english, russian]{babel}
% end for russian lang

% config page
%\usepackage[paperheight=6in,
%   paperwidth=5in,
%   top=10mm,
%   bottom=20mm,
%   left=10mm,
%   right=10mm]{geometry}

\usepackage[a4paper, margin=2cm]{geometry}
\usepackage{amsfonts}
\usepackage{amsmath}
\usepackage{amssymb}
\usepackage{indentfirst}
\usepackage{csquotes}

\title{Подготовка к экзамену ИиДУ}
\author{ИУ6-25Б}
\date{2024}

\begin{document}

\maketitle

\section*{1. Сформулировать определение первообразной. Сформулировать свойства первообразной и неопределённого интеграла. Сформулировать и доказать теорему об интегрировании по частям для неопределённого интеграла.}
\textbf{Опр.} Функция \(F(x)\) называется первообразной функции \(f(x)\) на интервале \((a, b)\), если \(F(x)\) диф-ма на \((a, b)\) и \(F'(x)=f(x) \; \forall x \in (a, b)\), где \((a, b)\) может быть любым.
\subsection*{Свойства первообразной:}
\begin{enumerate}
\item Если \(F(x)\) - первообразная \(f(x)\) на \((a, b)\), то \(F(x) + C\) тоже первообразная \(f(x)\) на \((a, b)\).
\item Если функция \(\Phi(x)\) диф-ма на \((a, b)\) и \(\Phi'(x) = 0 \; \forall x\in (a, b)\), то \(\Phi(x) = const\) на \((a, b)\).
\item Если \(F_1(x)\) и \(F_2(x)\) - первообразные \(f(x)\) на \((a, b)\), то \(F_1(x) - F_2(x) = C,\space C = const\).
\item Если функция \(f(x)\) непрерывна на \((a, b)\), то она имеет первообразную на этом интервале.
\end{enumerate}

\subsection*{Свойства неопределённого интеграла:}
\begin{enumerate}
\item \((\int f(x)dx)'=f(x)\)
\item \(d(\int f(x)dx)=f(x)dx\)
\item \(\int dF(x) = F(x) + C\)
\item \(\int (f_1(x) + ... + f_n(x))dx = \int f_1(x)dx + ... + \int f_n(x)dx + C\)
\item \(\int Af(x)dx = A\int f(x)dx + C\)
\item Если $\int f(x) \, dx = F(x) + C$ и $u = \varphi(x)$, то $\int f(u) \, du = F(u) + C$
\end{enumerate}

\subsection*{Доказательство теоремы:}
\par\textbf{Теорема.} (об интегрировании по частям)
\par Если $u(x)$ и $v(x)$ - гладкие функции, дифференцируемые множестве $\{x\}$, то $$\int u \, dv = uv - \int v \, du$$
\par\textbf{Доказательство.}
\par $d(uv) = udv + vdu \Rightarrow udv = d(uv) - vdu$
\par $u(x)$ и $v(x)$ - непрерывны на $[a, b] \Rightarrow \exists$ определённый интеграл от функций:
$$\int u \, dv = \int d(uv) - \int v \, du \Rightarrow \int u\, dv = uv - \int v \, du \Large_{\blacktriangle}$$


\section*{2. Разложение правильной рациональной дроби на простейшие. Интегрирование простейших дробей.}
\textbf{Опр.} Рациональной дробью называется дробь вида \(R(x) = \frac{Q_{m}(x)}{P_{n}(x)}\), где \(Q_{m}(x)\) и \(P_{n}(x)\) - многочлены от \(x\) степени \(m\) и \(n\) соответственно.\par
\textbf{Опр.} Дробь называется правильной, если степень числителя меньше степени знаменателя: \(m < n\), и неправильной, если \(m > n\).\par
\textbf{Опр.} Простейшими дробями называются дроби:
\begin{enumerate}
\item {\Large \(\frac{A}{x-a}\)} - I тип
\item {\Large \(\frac{A}{(x-a)^k}\)}, \(k \in \mathbb{Z} , k > 1\) - II тип
\item {\Large \(\frac{Mx + N}{x^2 + px + q}\)} - III тип
\item {\Large \(\frac{Mx + N}{(x^2 + px + q)^k}\)}, \(k \in \mathbb{Z}, k > 1\) - IV тип
\end{enumerate}

\par
\textbf{Теорема.} (о разложении правильной рациональной дроби в сумму простейших)
Правильная рациональная дробь {\Large \(\frac{Q_m(x)}{P_n(x)}\)}, \(m < n\),
где \(P_n(x) = a_{0}(x - x_{1})^{k_{1}} \dots (x - x_{s})^{k_{s}}(x^2 + p_{1}x + q_{1})^{l_{1}} \dots (x^2 + p_{m}x + q_{m})^{l_{m}}\),
единственным образом может быть представлена в виде суммы элементарных дробей:
\[\frac{Q_m(x)}{P_n(x)} = \frac{1}{a_{0}} \left( \frac{A_{1}}{(x - x_{1})^{k_{1}}} + \frac{A_{2}}{(x - x_{1})^{k_{1}-1}} + \dots + \frac{A_{k_{1}}}{(x - x_{1})^{1}} + \right.\]
\[+ \frac{B_{1}}{(x - x_{s})^{k_{s}}} + \frac{B_{2}}{(x - x_{s})^{k_{s}-1}} + \dots + \frac{B_{k_{s}}}{(x - x_{s})^{1}} + \dots  +\]
\[+ \frac{C_{1}x + D_{1}}{(x^2 + xp_{1} + q_{1})^{l_{1}}} + \frac{C_{2}x + D_{2}}{(x^2 + xp_{2} + q_{2})^{l_{1}-1}} + \dots + \frac{C_{l_{1}}x + D_{l_{1}}}{(x^2 + xp_{l_{1}} + q_{l_{1}})^{1}} + \dots + \]
\[\left. \frac{M_{1}x + N_{1}}{(x^2 + xp_{m} + q_{m})^{l_{m}}} + \frac{M_{2}x + N_{2}}{(x^2 + xp_{m} + q_{m})^{l_{m}-1}} + \dots + \frac{M_{l_{m}}x + N_{l_{m}}}{(x^2 + xp_{m} + q_{m})^{1}} \right)\]
\subsection*{Интегрирование простейших дробей:}
\begin{itemize}
\item I тип:
\[\int \frac{A}{x - a} \, dx = A \int \frac{d(x - a)}{x - a} = A \ln |x - a| + C\]
\item II тип: \[\int \frac{A}{(x - a)^k} \, dx = A \int (x - a)^{-k} \, d(x - a) = A \frac{{(x - a)^{-k+1}}}{-k + 1} + C = \frac{A}{1 - k} \frac{1}{(x-a)^{k-1}} + C\]
\item III тип: \[\int \frac{{Mx + N}}{x^2 + px + q} \, dx = \dots\]
\item IV тип: \[\int \frac{{Mx + N}}{(x^2 + px + q)^k} \, dx = \dots\]
\end{itemize}

\section*{3. Сформулировать свойства определенного интеграла. Доказать теорему о сохранении определенным интегралом знака подынтегральной функции.}
\subsection*{Свойства определённого интеграла:}
\par\textbf{Теорема 1.} Определённый интеграл алгебраической суммы интегрируемых на $[a, b]$ функций равен алгебраической сумме интегралов от слагаемых: $$\int_{a}^b f_{1}(x) \pm f_{2}(x) \pm \dots \pm f_{n}(x)\, dx = \int_{a}^b f_{1}(x) \, dx \pm \int_{a}^b f_{2}(x) \, dx  \pm \dots \pm \int_{a}^b f_{n}(x) \, dx$$
\par\textbf{Теорема 2.} Если $f(x)$ интегрируема на $[a, b]$, то $$\int_{a}^b c f(x) \, dx = c \int_{a}^b f(x) \, dx, \; c = const$$
\par\textbf{Теорема 3.} $$\int_{a}^b c \, dx = c \int_{a}^b \, dx = c(b - a)$$
\par\textbf{Теорема 4.} $$\int_{a}^b f(x) \, dx = - \int_{b}^a f(x) \, dx$$
\par\textbf{Теорема 5.} Если функция $y = f(x)$ интегрируема на $[a, b]$ и $f(x) \geq 0 \; (f(x) \leq 0) \; \forall x \in [a, b]$, то $$\int_{a}^b f(x) \, dx \geq 0 \; \left( \int_{a}^b f(x) \, dx \leq 0 \right)$$
\par\textbf{Теорема 6.} Для любых чисел $a, b, c$, расположенных в интервале интегрируемости функции $f(x)$ справедливо равенство (при условии, что все эти 3 интервала существуют): $$\int_{a}^b f(x) \, dx = \int_{a}^c f(x) \, dx + \int_{c}^b f(x) \, dx$$
\par\textbf{Теорема 7.} (об интегрировании неравенства)
\par Если функции $f(x)$ и $g(x)$ интегрируемы на $[a, b]$ и $f(x) \geq g(x) \; \forall x \in [a, b] \; (f(x) \neq g(x))$, то $$\int_{a}^b f(x) \, dx \geq \int_{a}^b g(x) \, dx$$
\par\textbf{Теорема 8.} (об оценке модуля определённого интеграла)
\par Если функция $f(x)$ непрерывна на $[a, b]$, то $$\left| \int_{a}^b f(x) \, dx \right| \leq \int_{a}^b |f(x)| \, dx$$
\par\textbf{Теорема 9.} (об оценке определённого интеграла)
\par Если $m$ и $M$ соответственно наименьшее и наибольшее значения интегрируемой на $[a, b]$ функции $f(x)$, то $$m(b - a) \leq \int_{a}^b f(x) \, dx \leq M(b - a)$$
\par\textbf{Теорема 10.} (об инвариантности неравенства)
\par Если $m$ и $M$ соответственно наименьшее и наибольшее значения интегрируемой на $[a, b]$ функции $f(x)$ и функция $\varphi(x) \geq 0$ и интегрируема на $[a, b]$, то $$m \int_{a}^b \varphi(x) \, dx \leq \int_{a}^b f(x) \varphi(x) \, dx \leq M \int_{a}^b \varphi(x) \, dx$$
\par\textbf{Теорема 11.} (о среднем)
\par Если функция $f(x)$ непрерывна на $[a, b]$, a функция $\varphi(x)$ интегрируема и знакопостоянна на $[a, b]$, то $\exists$ точка $c \in (a, b)$ такая, что $$\int_{a}^b f(x)\varphi(x) \, dx = f(c) \int_{a}^b \varphi(x) \, dx$$
\subsection*{Доказательство теоремы:}
\par\textbf{Теорема о сохранении интегралом знака подынтегральной функции(теорема 5).}
\par Если функция $y = f(x)$ интегрируема на $[a, b]$ и $f(x) \geq 0 \; (f(x) \leq 0) \; \forall x \in [a, b]$, то $$\int_{a}^b f(x) \, dx \geq 0 \; \left( \int_{a}^b f(x) \, dx \leq 0 \right)$$
\par\textbf{Доказательство.}
\par Пусть $a < b$: $$\int_{a}^b f(x) \, dx = \lim_{ \substack{n \to \infty \\ \text{max}_{k}\Delta x \to 0}} \sum_{k = 1}^n f(\xi_{k}) \Delta x_{k}\text{ , где } \Delta x_{k} = x_{k} - x_{k - 1}$$
\par Пусть $f(x) \geq 0 \; \forall x \in [a, b]$, тогда $f(\xi_{k}) \geq 0, \Delta x_{k} > 0 \Rightarrow f(\xi_{k}) \Delta x_{k} \geq 0 \; \forall k$, тогда: $$\sum_{k=1}^n f(\xi_{k}) \Delta x_{k} \geq 0 \Rightarrow \lim_{ \substack{n \to \infty \\ \text{max}_{k}\Delta x \to 0}} \sum_{k = 1}^n f(\xi_{k}) \Delta x_{k} \geq 0 \Rightarrow \int_{a}^b f(x) \, dx \geq 0 _\blacktriangle$$

\section*{4. Сформулировать свойства определенного интеграла. Доказать теорему об оценке определенного интеграла.}
\par\textbf{Свойства определённого интеграла см. в пункте 3.}
\par\textbf{Теорема об оценке определённого интеграла (теорема 9).}
\par Если $m$ и $M$ соответственно наименьшее и наибольшее значения интегрируемой на $[a, b]$ функции $f(x)$, то $$m(b - a) \leq \int_{a}^b f(x) \, dx \leq M(b - a)$$
\par\textbf{Доказательство.}
\par По условию $m \leq f(x) \leq M$, где $m = min_{[a, b]} f(x), M = max_{[a, b]} f(x)$, и $f(x)$ интегрируема на $[a, b]$, тогда
$$\int_a^b  m \, dx \leq \int_{a}^b f(x)\, dx  \leq \int_{a}^b M \, dx \Rightarrow$$
$$\Rightarrow m(b - a) \leq \int_{a}^b f(x)\, dx \leq M(b - a) _\blacktriangle$$

\section*{5. Сформулировать свойства определенного интеграла. Доказать теорему об оценке модуля определенного интеграла.}
\par\textbf{Свойства определённого интеграла см. в пункте 3.}
\par\textbf{Теорема об оценке модуля определённого интеграла (теорема 8).}
\par Если функция $f(x)$ непрерывна на $[a, b]$, то $$\left| \int_{a}^b f(x) \, dx \right| \leq \int_{a}^b |f(x)| \, dx$$
\par\textbf{Доказательство.}
\par $f(x)$ непрерывна на $[a, b] \Rightarrow -|f(x)| \leq f(x) \leq |f(x)|$
\par По теореме 7: $$- \int_{a}^b |f(x)| \, dx \leq \int_{a}^b f(x) \, dx \leq \int_{a}^b |f(x)| \, dx$$
\par Тогда по определению модуля: $$\left| \int_{a}^b f(x) \, dx \right| \leq \int_{a}^b |f(x)| \, dx _{\blacktriangle}$$

\section*{6. Сформулировать свойства определенного интеграла. Доказать теорему о среднем для определенного интеграла.}
\par\textbf{Свойства определённого интеграла см. в пункте 3.}
\par\textbf{Теорема о среднем (теорема 11).}
\par Если функция $f(x)$ непрерывна на $[a, b]$, a функция $\varphi(x)$ интегрируема и знакопостоянна на $[a, b]$, то $\exists$ точка $c \in (a, b)$ такая, что $$\int_{a}^b f(x)\varphi(x) \, dx = f(c) \int_{a}^b \varphi(x) \, dx$$
\par\textbf{Доказательство.}
\par $f(x)$ непрерывна на $[a, b] \Rightarrow$ по теореме Вейерштрасса она достигает на  $[a, b]$ наименьшее и наибольшее значения $m = min_{[a, b]} f(x)$ и $M = max_{[a, b]} f(x)$ и $m \leq f(x) \leq M \quad \forall x \in [a, b]$
\par Пусть $\varphi > 0$:
$m \varphi(x) \leq f(x) \varphi(x) \leq M \varphi(x)$
$$m \int_{a}^b \varphi(x) \, dx  \leq \int_{a}^b f(x) \varphi(x) \, dx \leq M \int_{a}^b \varphi(x) \, dx$$
\par Так как $\varphi(x) > 0$, то $\int_{a}^b \varphi(x) \, dx \geq 0$, тогда $m \leq  \frac{{\int_{a}^b f(x) \varphi(x) \, dx}}{\int_{a}^b \varphi(x) \, dx} \leq M$
\par По теореме Больцано-Коши $\exists \, c \in (a, b)$ такая, что:
$$f(c) = \frac{{\int_{a}^b f(x) \varphi(x) \, dx}}{\int_{a}^b \varphi(x) \, dx} \Rightarrow \int_{a}^b f(x) \varphi(x) \, dx = f(c) \int_{a}^b \varphi(x) \, dx, \text{ где } c \in (a, b) _{\blacktriangle}$$

\section*{7. Сформулировать определение интеграла с переменным верхним пределом. Доказать теорему о производной от интеграла с переменным верхним пределом.}
\par\textbf{Опр.} Функция $ {Y(x) = \int_a^x f(t) \, dt}$, определённая на $[a, b]$, называется определённым интегралом с переменным верхним пределом, где $[a, x] \subset [a, b]$
\par\textbf{Теорема.} (о производной от интеграла с переменным верхним пределом)
\par Если функция $f(x)$ интегрируема на $[a, b]$ и непрерывна на нём, то $$Y'(x) = \frac{d}{dx} \int_{a}^x f(t) \, dt = f(x)$$
\par\textbf{Доказательство.}
\par $Y(x) = \int_{a}^x f(t) \, dt$
\par $Y(x + \Delta x) = \int_{a}^{x + \Delta x} f(t) \, dt, (x + \Delta x) \in [a, b]$
\par $f(x)$ непрерывна на $[a, b]$, следовательно, по теореме о среднем:
\par $\int_{x}^{x + \Delta x} f(t) \, dt = f(c) (x + \Delta x - x) = f(c) \Delta x$, где $c \in (x, x + \Delta x)$
\par По определению производной ($\Delta x \to 0, x < c < x + \Delta x \Rightarrow c \to x$):
$$Y'(x) = \lim_{\Delta x \to 0} \frac{{f(c) \Delta x}}{\Delta x} = \lim_{ c \to x } f(c) = f(x) _\blacktriangle$$

\section*{8. Сформулировать свойства определенного интеграла. Вывести формулу Ньютона-Лейбница.}
\par\textbf{Свойства определённого интеграла см. в пункте 3.}
\par\textbf{Теорема.} (формула Ньютона-Лейбница)
\par Если функция $y = f(x)$ непрерывна на $[a, b]$, то $$\int_{a}^b f(x) \, dx = F(b) - F(a)$$
\par\textbf{Доказательство.}
\par Пусть $F(x)$ - $\forall$ первообразная функции $f(x)$ на $[a, b]$
\par $Y(x) = \int_{a}^x f(x) \, dx$ - тоже первообразная функции $f(x)$ на $[a, b]$
\par Тогда по основной теореме о первообразных: $\int_{a}^x f(x) \, dx = F(x) + C, c =const \qquad(1)$
\par Положим $x = a$: $\int_{a}^a f(t) \, dt = F(a) + C \Rightarrow F(a) + C = 0 \Rightarrow C = -F(a)$
\par Подставим в $(1)$ и получим: $\int_{a}^x f(t) \, dt = F(x) - F(a)$
\par Положим $x = b$: $$\int_{a}^b f(t) \, dt = F(b) - F(a) _{\blacktriangle}$$

\section*{9. Дать геометрическую интерпретацию определенного интеграла. Сформулировать и доказать теорему об интегрировании подстановкой для определенного интеграла.}
\par\textbf{Сделать рисунок.} Геометрическая интерпретация определённого интеграла - площадь криволинейной трапеции, ограниченной графиком функции, осью $Ox$ и прямыми $x = a, x = b$
\par\textbf{Теорема.} (о замене переменной в определённом интеграле)
\par Если функция $f(x)$ непрерывна на $[a, b]$, а функции $x = \varphi(t), \varphi'(t), f(\varphi(t))$ непрерывны на $[a, b]$ и $\varphi(\alpha) = a, \varphi(\beta) = b$, то $$\int_{a}^b f(x) \, dx = \int_{\alpha}^\beta f(\varphi(t)) \varphi'(t) \, dt$$
\par\textbf{Доказательство.}
\par Формулы замены переменной в неопределённом интеграле: $$\int f(x) \, dx = \int f(\varphi(t)) \varphi'(t) \, dt$$
\par Если $F(x)$ - первообразная функции $f(x)$, то $F(\varphi(t))$ - первообразная функции $f(\varphi(t))\varphi'(t)$
\par По формуле Ньютона-Лейбница: $$\int_{a}^b f(x) \, dx = F(x)|_{a}^b = F(b) - F(a)$$
\par Так как по условию: $\varphi(\alpha) = a, \varphi(\beta) = b$:
$$\int_{\alpha}^\beta f(\varphi(t)) \varphi'(t) \, dt = F(\varphi(t))|_{\alpha}^\beta = F(\varphi(\beta)) - F(\varphi(\alpha)) = F(b) - F(a)$$
\par Получим: $$\int_{a}^b f(x) \, dx = \int_{\alpha}^\beta f(\varphi(t)) \varphi'(t) \, dt _{\blacktriangle}$$

\section*{10. Сформулировать свойства определенного интеграла. Интегрирование периодических функций. Интегрирование четных и нечетных функций на отрезке, симметричном относительно начала координат.}
\par\textbf{Свойства определённого интеграла см. в пункте 3.}
\par\textbf{Периодические функции:}
\par Функция $f(x)$ - периодическая с периодом $T$ и непрерывная на $[a, a + T]$ $$\int_{a}^{a + T} f(x) \, dx = \int_{0}^{T} f(x) \, dx$$
\par\textbf{Чётные функции:}
\par Функция $f(x)$ - чётная на $[-a, a]$, то есть $\forall x \in [-a, a] \; f(-x) = f(x)$: $$\int_{-a}^a f(x) \, dx = 2 \int_{0}^a f(x) \, dx$$
\par\textbf{Нечётные функции:}
\par Функция $f(x)$ - нечётная на $[-a, a]$, то есть $\forall x \in [-a, a] \; f(-x) = -f(x)$: $$\int_{-a}^a f(x) \, dx = 0$$

\section*{11. Сформулировать свойства определенного интеграла. Сформулировать и доказать теорему об интегрировании по частям для определённого интеграла.}
\par\textbf{Свойства определённого интеграла см. в пункте 3.}
\par\textbf{Теорема.} Если $u(x)$ и $v(x)$ - непрерывные функции, дифференцируемые в $(a, b)$, то $$\int_{a}^b u \, dv = uv\vert_{a}^b - \int_{a}^b v \, du$$
\par\textbf{Доказательство.}
\par $d(uv) = udv + vdu \Rightarrow udv = d(uv) - vdu$
\par $u(x)$ и $v(x)$ - непрерывны на $[a, b] \Rightarrow \exists$ определённый интеграл от функций: $$\int_{a}^b u \, dv = \int_{a}^b \, d(uv) - \int_{a}^b v\, du \Rightarrow \int_{a}^b u\, dv = uv \vert_{a}^b - \int_{a}^b v \, du _{\blacktriangle}$$

\section*{12. Сформулировать определение несобственного интеграла 1-го рода. Сформулировать и доказать признак сходимости по неравенству для несобственных интегралов 1-го рода.}
\par\textbf{Опр.} Пусть функция $y = f(x)$ определена и непрерывна для $\forall x \in [a, +\infty)$. Тогда несобственным интегралом первого рода $\int_{a}^{+\infty} f(x) \, dx$ называется предел определённого интеграла с переменным верхним пределом $\int_{a}^{b} f(x) \, dx$ при $b \to +\infty$:$$\int_{a}^{+\infty} f(x) \, dx = \lim_{ b \to \infty}{\int_{a}^b f(x) \, dx }$$
\par Аналогично для бесконечного нижнего предела интегрирования.

\par\textbf{Теорема.} (признак сходимости по неравенству для несобственных интегралов 1-ого рода)
\par Если функции $f(x)$ и $\varphi(x)$ непрерывны на $[a, +\infty)$ и выполняется неравенство $0 < f(x) \leq \varphi(x) \; \forall x \in [a, +\infty)$, тогда:
\begin{enumerate}
\item если сходится $\int_{a}^{+\infty} \varphi(x) \, dx$, то $\int_{a}^{+\infty} f(x) \, dx$ тоже сходится
\item если расходится $\int_{a}^{+\infty} f(x) \, dx$, то $\int_{a}^{+\infty} \varphi(x) \, dx$ тоже расходится
\end{enumerate}
\par\textbf{Доказательство.}
\par\textbf{1)} По условию $\int_{a}^{+\infty} \varphi(x) \, dx$ сходится, $\implies$, $\exists$ конечный предел $$\lim_{b \to +\infty}{\int_{a}^b \varphi(x) \, dx } = M \implies \int_{a}^{b} \varphi(x) \, dx \leq M$$
\par По условию $\forall x \in [a, +\infty) \; 0 < f(x) \leq \varphi(x)$, тогда по теореме об интегрировании неравенства  $0 < \int_{a}^b f(x) \, dx \leq \int_{a}^{b} \varphi(x) \, dx \leq M$
\par Пусть $b_{1} \in (b, +\infty)$. Рассмотрим:
$$\int_{a}^{b_{1}} f(x) \, dx = \int_{a}^{b} f(x) \, dx + \int_{b}^{b_{1}} f(x) \, dx > \int_{a}^{b} f(x) \, dx \implies$$
$$\implies \int_{a}^{b} f(x) \, dx \text{ есть функция, возрастающая с возрастанием } b$$
\par Тогда по теореме Вейерштрасса: $$\exists \lim_{ b \to +\infty } \int_{a}^{b} f(x) \, dx \leq M \implies \int_{a}^{+\infty} f(x) \, dx \text{ - сходящийся}$$
\par\textbf{2)} (от противного)
\par Предположим, что $\int_{a}^{+\infty} \varphi(x) \, dx$ сходится, тогда по доказательству \textbf{1)} $\int_{a}^{+\infty} f(x) \, dx$ тоже сходится, что противоречит условию, $\implies$, $\int_{a}^{+\infty} \varphi(x) \, dx$ расходится $_\blacktriangle$

\section*{13. Сформулировать определение несобственного интеграла 1-ого рода. Сформулировать и доказать предельный признак сравнения для несобственных интегралов 1-ого рода.}
\par\textbf{Опр.} Пусть функция $y = f(x)$ определена и непрерывна для $\forall x \in [a, +\infty)$. Тогда несобственным интегралом первого рода $\int_{a}^{+\infty} f(x) \, dx$ называется предел определённого интеграла с переменным верхним пределом $\int_{a}^{b} f(x) \, dx$ при $b \to +\infty$:$$\int_{a}^{+\infty} f(x) \, dx = \lim_{ b \to \infty}{\int_{a}^b f(x) \, dx }$$
\par Аналогично для бесконечного нижнего предела интегрирования.
\par\textbf{Теорема.} (предельный признак сравнения для несобственных интегралов 1-ого рода)
\par Пусть функции $y = f(x)$ и $y = g(x)$ интегрируемы на отрезке $[a, b] \subset [a, +\infty)$, $f(x) \geq 0, g(x) > 0 \; \forall x \geq a$ и существует конечный предел $\lim_{ x \to +\infty }{\frac{f(x)}{g(x)}} = \lambda (\not = 0)$. Тогда несобственные интегралы $\int_{a}^{+\infty} f(x) \, dx$ и $\int_{a}^{+\infty} g(x) \, dx$ сходятся или расходятся одновременно.
\par\textbf{Доказательство.}
\par По условию и определению предела:$$\lim_{ x \to +\infty }{\frac{f(x)}{g(x)}} = \lambda \iff \forall \varepsilon > 0 \; \exists M(\varepsilon) > 0: \forall x > M \Rightarrow \left| \frac{f(x)}{g(x)} - \lambda \right| < \varepsilon$$
\par Рассмотрим неравенство:
$$- \varepsilon < \frac{f(x)}{g(x)} - \lambda < \varepsilon$$
$$- \varepsilon + \lambda < \frac{f(x)}{g(x)} < \varepsilon + \lambda$$
$$(\lambda - \varepsilon) g(x) < f(x) < (\lambda + \varepsilon) g(x) \; \forall x > M \qquad (*)$$
\par Проинтегрируем правую часть:$$\int_{a}^{+\infty} f(x) \, dx < (\lambda + \varepsilon) \int_{a}^{+\infty} g(x) \, dx$$
\begin{enumerate}
\item Пусть $\int_{a}^{+\infty} g(x) \, dx$ сходится, тогда $(\lambda + \varepsilon) \int_{a}^{+\infty} g(x) \, dx$ тоже сходится, так как $(\lambda + \varepsilon)$ - число, не влияющее на сходимость. По теореме о признаке сходимости по неравенству несобственных интегралов 1-ого рода $\int_{a}^{+\infty} f(x) \, dx$ тоже сходится.
\item Пусть $\int_{a}^{+\infty} f(x) \, dx$ расходится, тогда по теореме о признаке сходимости по неравенству несобственных интегралов 1-ого рода $(\lambda + \varepsilon) \int_{a}^{+\infty} g(x) \, dx$ тоже расходится.
\end{enumerate}
\par Аналогично, интегрируя левую часть неравенства $(*)$ получим:
\begin{enumerate}
\item[3.] Если $\int_{a}^{+\infty} f(x) \, dx$ сходится, то $\int_{a}^{+\infty} g(x) \, dx$ тоже сходится.
\item[4.] Если $\int_{a}^{+\infty} g(x) \, dx$ расходится, то $\int_{a}^{+\infty} f(x) \, dx$ тоже расходится.
\end{enumerate}
\par В итоге получим, что интегралы $\int_{a}^{+\infty} f(x) \, dx$ и $\int_{a}^{+\infty} g(x) \, dx$ сходятся или расходятся одновременно. $_\blacktriangle$

\section*{14. Сформулировать определение несобственного интеграла 1-ого рода. Сформулировать и доказать признак абсолютной сходимости для несобственных интегралов 1-ого рода.}
\par\textbf{Опр.} Пусть функция $y = f(x)$ определена и непрерывна для $\forall x \in [a, +\infty)$. Тогда несобственным интегралом первого рода $\int_{a}^{+\infty} f(x) \, dx$ называется предел определённого интеграла с переменным верхним пределом $\int_{a}^{b} f(x) \, dx$ при $b \to +\infty$:$$\int_{a}^{+\infty} f(x) \, dx = \lim_{ b \to \infty}{\int_{a}^b f(x) \, dx }$$
\par Аналогично для бесконечного нижнего предела интегрирования.
\par\textbf{Теорема.} (признак абсолютной сходимости для несобственных интегралов 1-ого рода)
\par Если функция $f(x)$ непрерывна и знакопеременна на $[a, +\infty)$ и $\int_{a}^{+\infty} |f(x)| \, dx$ сходится, то $\int_{a}^{+\infty} f(x) \, dx$ сходится.
\par\textbf{Доказательство.}
\par $f(x)$ непрерывна на $[a, +\infty)$ (по условию), $\implies$, $\forall x \in [a, +\infty)$ справедливо неравенство: $$-|f(x)| \leq f(x) \leq |f(x)| \implies 0 \leq f(x) + |f(x)| \leq 2 |f(x)|$$
$$\int_{a}^{+\infty} |f(x)| \, dx \text{ сх-ся(по усл.)} \implies 2\int_{a}^{+\infty} |f(x)| \, dx \text{ сх-ся(св-во линейности)} \qquad (1)$$
$$f(x) + |f(x)| \leq 2 |f(x)| \; \forall x \in [a, +\infty) \qquad (2)$$
\par Из $(1)$ и $(2)$: $$\int_{a}^{+\infty} (f(x) + |f(x)|) \, dx \text{ сх-ся (по 1 признаку сравнения по нер-ву)}$$
\par Тогда:$$\int_{a}^{+\infty} f(x) \, dx = \int_{a}^{+\infty} (f(x) + |f(x)|) \, dx - \int_{a}^{+\infty} |f(x)| \, dx$$
\par Оба слагаемых сходятся, $\implies$, $\int_{a}^{+\infty} f(x) \, dx$ сходится$_{\blacktriangle}$

\section*{15. Сформулировать определение несобственного интеграла 2-ого рода и признаки сходимости таких интегралов. Сформулировать и доказать признак абсолютной сходимости для несобственных интегралов 1-ого рода.}
\par\textbf{Опр.} Несобственный интеграл второго рода от функции $f(x)$, непрерывной на $[a, b)$ и неограниченной в окрестности точки $b$, называется сходящимся, если **существует конечный предел** при $\epsilon \to +0$ определённого интеграла $\int_{a}^{b-\varepsilon} f(x) \, dx$:$$\int_{a}^{b} f(x) \, dx = \lim_{ \varepsilon \to +0}{\int_{a}^{b - \varepsilon} f(x) \, dx }$$
\par Аналогично для функции, неограниченной в окрестности точки $a$.
\par\textbf{Теорема.} Если функции $f(x)$ и $\varphi(x)$ непрерывны на $[a, b)$ и выполняется неравенство $0 < f(x) \leq \varphi(x) \; \forall x \in [a, b)$, тогда:
\begin{enumerate}
\item если сходится $\int_{a}^{b} \varphi(x) \, dx$, то $\int_{a}^{b} f(x) \, dx$ тоже сходится
\item если расходится $\int_{a}^{b} f(x) \, dx$, то $\int_{a}^{b} \varphi(x) \, dx$ тоже расходится
\end{enumerate}
\par\textbf{Теорема.} Пусть функции $y = f(x)$ и $y = g(x)$ интегрируемы на $[a, b)$, $f(x) \geq 0, g(x) > 0 \; \forall x \geq a$ и существует конечный предел $\lim_{ x \to b }{\frac{f(x)}{g(x)}} = \lambda (\not = 0)$. Тогда несобственные интегралы $\int_{a}^{b} f(x) \, dx$ и $\int_{a}^{b} g(x) \, dx$ сходятся и расходятся одновременно.
\par\textbf{Теорема.} Если функция $f(x)$ непрерывна и знакопеременна на $[a, b)$ и $\int_{a}^{b} |f(x)| \, dx$ сходится, то $\int_{a}^{b} f(x) \, dx$ сходится.
\subsection*{Доказательство теоремы:}
\par\textbf{Теорема.} (признак абсолютной сходимости для несобственных интегралов 1-ого рода)
\par Если функция $f(x)$ непрерывна и знакопеременна на $[a, +\infty)$ и $\int_{a}^{+\infty} |f(x)| \, dx$ сходится, то $\int_{a}^{+\infty} f(x) \, dx$ сходится.
\par\textbf{Доказательство.}
\par $f(x)$ непрерывна на $[a, +\infty)$ (по условию), $\implies$, $\forall x \in [a, +\infty)$ справедливо неравенство: $$-|f(x)| \leq f(x) \leq |f(x)| \implies 0 \leq f(x) + |f(x)| \leq 2 |f(x)|$$
$$\int_{a}^{+\infty} |f(x)| \, dx \text{ сх-ся(по усл.)} \implies 2\int_{a}^{+\infty} |f(x)| \, dx \text{ сх-ся(св-во линейности)} \qquad (1)$$
$$f(x) + |f(x)| \leq 2 |f(x)| \; \forall x \in [a, +\infty) \qquad (2)$$
\par Из $(1)$ и $(2)$: $$\int_{a}^{+\infty} (f(x) + |f(x)|) \, dx \text{ сх-ся (по 1 признаку сравнения по нер-ву)}$$
\par Тогда:$$\int_{a}^{+\infty} f(x) \, dx = \int_{a}^{+\infty} (f(x) + |f(x)|) \, dx - \int_{a}^{+\infty} |f(x)| \, dx$$
\par Оба слагаемых сходятся, $\implies$, $\int_{a}^{+\infty} f(x) \, dx$ сходится$_{\blacktriangle}$


\section*{16. Фигура, ограниченная кривой $y = f(x) \geq 0$ и прямыми $x = a, x = b$ и $y = 0 \; (a < b)$. Вывести формулу для вычисления с помощью определённого интеграла площади этой фигуры.}
\par\textbf{Сделать рисунок}
\par Рассмотрим криволинейную трапецию, ограниченную кривой $y = f(x) \geq 0$, прямыми $x = a, x = b$  и $y = 0$.
\par Отрезок $[a, b]$ оси $Oy$ - основание криволинейной трапеции.
\par Разобьём его на $n$ частичных отрезков точками $a = x_{0}, x_{1}, \dots, x_{n} = b$, где $x_{1} < x_{2} < \dots < x_{n}$
\par Через точки деления проведём прямые $|| \, Oy$, то есть исходную трапецию разобьём на $n$ трапеций.
\par Пусть $\xi_{k} \in [x_{k-1}, x_{k}], k = \overline{1, n}$
\par Составим сумму ($\Delta x_{k} = x_{k} - x_{k - 1}$):
$$\sum_{k=1}^{n} f(\xi_{k}) \Delta x_{k} \text{ - интегральная сумма Римана}$$
\par где $S_{k} = f(\xi_{k}) * \Delta x_{k}$ - площадь $k$-ого прямоугольника
$$S_{n} = \sum_{k = 1}^{n} S_{k} = \sum_{k=1}^{n} f(\xi_{k}) \Delta x_{k} \text{ - площадь ступенчатой фигуры}$$
\par Будем считать $S_{n}$ приближённым значением площади криволинейной трапеции. Тогда чем больше $n$ и чем меньше $\Delta x_{k}$, тем более точным будет это приближение. То есть:$$S = \lim_{ n \to \infty } S_{n} = \lim_{ \substack{n \to \infty \\ max_{k} \Delta x_{k} \to 0} } \sum_{k = 1}^{n} f(\xi_{k}) \Delta x_{k} = \int_{a}^b f(x) \, dx$$

\section*{17. Фигура ограничена лучами $\varphi = \alpha, \varphi = \beta$ и кривой $r = f(\varphi)$. Здесь $r$ и $\varphi$ - полярные координаты точки, $0 \leq \alpha < \beta \leq 2 \pi$. Вывести формулу для вычисления с помощью определённого интеграла площади этой фигуры.}
\par\textbf{Сделать рисунок}
\par Пусть дана непрерывная на $[\alpha, \beta]$ функция  $\rho = \rho(x)$ и $0 \leq \alpha \leq \varphi \leq \beta \leq 2\pi$
\par Разобьём криволинейный сектор лучами на $n$ криволинейных секторов:
\par $\alpha = \varphi_{0} < \varphi_{1} < \dots < \varphi_{k-1} < \varphi_{k} < \dots < \varphi_{n} = \beta$
\par $\Delta \varphi_{k} = \varphi_{k} - \varphi_{k-1}$
\par В каждом частичном секторе возьмём произвольно $\tilde\varphi_{k}, k = \overline{1, n}$, то есть $\tilde{\varphi_{k}} \in [\varphi_{k-1}, \varphi_{k}]$
\par $\rho(\tilde{\varphi_{k}})$ - радиус вектор, соответствующий углу $\tilde{\varphi_{k}}$
\par Площадь криволинейного сектора $\approx$ площадь кругового сектора
$$S_{n} = \sum_{k=1}^n S_{k} = \frac{1}{2} \sum_{k=1}^n \rho^2(\tilde{\varphi_{k}}) \Delta \varphi_{k} \text{ - интегральная сумма функции } \rho^2(\varphi)$$
\par $\rho = \rho(x)$ непрерывна на $[\alpha, \beta] \implies \rho^2(\varphi)$ тоже непрерывна на $[\alpha, \beta] \implies \exists$ конечный предел:$$\lim_{\substack{n \to \infty \\ \max_{k} \Delta \varphi_{k} \to 0}} \frac{1}{2} \sum_{k=1}^n \rho^2(\varphi) \Delta \varphi_{k} = \frac{1}{2} \int_{\alpha}^{\beta} \rho^2(\varphi)\, d\varphi$$
\par Итак:$$S_{n} = \frac{1}{2} \int_{\alpha}^{\beta} \rho^2(\varphi)\, d\varphi$$

\section*{18. Тело образованно вращением вокруг оси $Ox$ криволинейной трапеции, ограниченной кривой $y = f(x) \geq 0$, прямыми $x = a, x = b$ и $y = 0, a < b$. Вывести формулу для вычисления с помощью определённого интеграла объёма тела вращения.}
\par\textbf{Сделать рисунок}
\par Дано тело вращения
\par Пусть $S(x)$ - площадь поперечного сечения плоскостью $\perp Ox$, $a \leq x \leq b$, и $S(x)$ - непрерывная функция на $[a, b]$
\par Проведём плоскости $x = x_{0} = a, x = x_{1}, \dots, x = x_{n} = b$, они разбивают тело на слои
\par Выберем в каждом интервале точку $\xi_{k} \in (x_{k - 1}, x_{k}), k = \overline{1, n}$
\par Для каждого значения $\xi_{k}$ построим цилиндрическое тело, образующие которого $\perp Ox$, а направляющая есть контур сечения тела плоскостью $x = \xi_{k}$
\par Объём такого цилиндра $V_{k} = S(\xi_{k}) \Delta x_{k}$
\par Сложим все такие цилиндры:$$V_{n} = \sum_{k = 1}^{n} S(\xi_{k}) \Delta x_{k}$$
\par Получили приближённое значение объёма тела вращения, при увеличении $n$ и уменьшении $\Delta x_{k}$ приближение становится более точным. То есть: $$V = \lim_{ \substack{n \to \infty \\ max_{k} \Delta x_{k} \to 0} } \sum_{k = 1}^{n} S(\xi_{k}) \Delta x_{k} = \int_{a}^{b} S(x) \, dx \text{, где } S(x) \text{ - площадь поперечного сечения}$$
\par Если кривая задана $y = f(x)$, то сечения - окружности, площадь которых $S(x) = \pi y^2 = \pi f^2(x)$
\par Подставим в формулу объёма: $$V = \int_{a}^{b} \pi y^2\, dx = \pi \int_{a}^{b} f^2(x) \, dx$$

\section*{19. Кривая задана в декартовых координатах уравнение $y = f(x)$, где $x$ и $y$ - декартовы координаты точки, $a \leq x \leq b$. Вывести формулу для вычисления длины дуги этой кривой.}
\par\textbf{Сделать рисунок}
\par Пусть кривая $y = f(x)$, где $f(x)$ - непрерывная функция на $[a, b]$ и имеющая непрерывную первую производную на этом отрезке. Тогда $$l = \int_{a}^{b} \sqrt{ 1 + (f'(x))^2 } \, dx$$
\par Покажем это:
\par Разобьём дугу $AB$ на $n$ частей точками $M_{0}, M_{1}, \dots, M_{n}$, абсциссы которых $a = x_{0} < x_{1} < \dots < x_{n-1} < x_{n} = b$
\par Проведём хорды, соединив соседние точки, и получим ломанную, вписанную в дугу $AB$, эта ломаная состоит из отрезков $M_{0}M_{1}, M_{1}M_{2}, \dots, M_{n-1}M_{n}$, где $M_{0} = A, M_{n} = B$
\par Обозначим их за $l_{1}, l_{2}, \dots, l_{n}$: $l_{i} = M_{i-1}M_{i}$
\par Периметр этой ломаной $l_{n} = \sum_{k=1}^{n} l_{k}$
\par С уменьшением длин хорд ломаная по своей форме приближается к дуге $AB$
\begin{displayquote}
\par\textbf{Опр.} Длиной $l$ дуги $AB$ кривой $y = f(x)$ называется предел длины вписанной в неё ломаной, когда число её звеньев неограниченно растёт, а наибольшая из длин звеньев стремится к нулю: $$l = \lim_{\substack{n \to \infty \\ \max l_{k} \to 0}} \sum_{k=1}^n l_{k} \qquad(1)$$
\par При этом предположим, что этот предел существует и не зависит от выбора точек.
\par\textbf{Опр.} Кривые, для которых предел $(1)$ существует, называются спрямляемыми.
\end{displayquote}
\par По формуле расстояния между двумя точками на плоскости имеем:$$l_{k} = \sqrt{ (x_{k} - x_{k-1})^2 + (y_{k} - y_{k-1})^2 } = \sqrt{ (\Delta x_{k})^2 + (\Delta y_{k})^2} \text{, где}$$
\par $\Delta x_{k} = x_{k} - x_{k-1}$,
\par $\Delta y_{k} = y_{k} - y_{k-1} = f(x_{k}) - f(x_{k-1})$,
\par $y_{k} = f(x_{k})$,
\par $y_{k-1} = f(x_{k-1})$,
\par $k = \overline{1, n}$
$$l_{k} = \sqrt{ (\Delta x_{k})^2 + (\Delta y_{k})^2} = \Delta x_{k} \sqrt{ 1 + \left(\frac{\Delta y_{k}}{\Delta x_{k}}\right)^2 }$$
\par По теореме Лагранжа:$$\frac{\Delta y_{k}}{\Delta x_{k}} = \frac{f(x_{k}) - f(x_{k-1})}{x_{k} - x_{k-1}} = f'(\xi_{k}) \text{, где } x_{k-1} < \xi_{k} < x_{k}$$
\par Тогда $l_{k} = \sqrt{ 1 + (f'(\xi_{k}))^2 } \Delta x_{k}$ и длина вписанной ломаной: $$l_{n} = \sum_{k=1}^n \sqrt{ 1 + (f'(\xi_{k}))^2 } \Delta x_{k} \text{ - интегральная сумма}\qquad (*)$$
\par $f'(x)$ непрерывна на $[a, b]$, $\implies$, $\sqrt{ 1 + (f'(\xi_{k}))^2 } \Delta x_{k}$ тоже непрерывна на $[a, b]$, поэтому существует предел интегральной суммы $(*)$, который равен определённому интегралу:$$l = \lim_{\substack{n \to \infty \\ \max_{k} \Delta x_{k} \to 0}} \sum_{k=1}^n \sqrt{ 1 + (f'(\xi_{k}))^2 } \Delta x_{k} = \int_{a}^b \sqrt{ 1 + (f'(x))^2 } \, dx$$
\par Получили формулы для вычисления длины дуги кривой в декартовых координатах:$$l = \int_{a}^b \sqrt{ 1 + (f'(x))^2 } \, dx$$

\section*{20. Кривая задана в полярных координатах уравнением $r = f(\varphi) \geq 0$, где $r$ и $\varphi$ - полярные координаты точки, $\alpha \leq \varphi \leq \beta$. Вывести формулу для вычисления длины дуги этой кривой.}
\par\textbf{Сделать рисунок}
\par Имеем: $$\left\{\begin{array}{@{}l@{}} x = r \cos \varphi \\ y = r \sin \varphi \end{array}\right.\,$$
\par Найдём:
\par $x_{\varphi}' = r' \cos \varphi - r \sin \varphi$
\par $y_{\varphi}' = r' \sin \varphi + r \cos \varphi$
\par Используем формулу длины дуги графика функции, заданной параметрически:$$l = \int_{\varphi_{1}}^{\varphi_{2}} \sqrt{ (x_{\varphi}')^2 + (y_{\varphi}')^2 } \, d\varphi = \int_{\varphi_{1}}^{\varphi_{2}} \sqrt{ (r' \cos \varphi - r \sin \varphi)^2 + (r' \sin \varphi + r \cos \varphi)^2 }\, d\varphi =$$
$$= \int_{\varphi_{1}}^{\varphi_{2}} \sqrt{ (r')^2 \cos^2 \varphi - 2 r r' \sin \cos + r^2\sin^2 \varphi + (r')^2 \sin^2 \varphi + 2 r r' \sin \cos + r^2 \cos^2 \varphi } \, d\varphi$$
$$l = \int_{\varphi_{1}}^{\varphi_{2}} \sqrt{ (r')^2 + r^2 }\, d\varphi$$

\section*{21. Линейные дифференциальные уравнения первого порядка. Интегрирование линейных неоднородных дифференциальных уравнений первого порядка методом Бернулли (метод "$u \cdot v$") и методом Лагранжа (вариации произвольной постоянной).}
\par\textbf{Опр.} Линейным ДУ первого порядка называется ДУ, линейное относительно неизвестной функции и её первой производных, т.е. ДУ вида:
$$y' + p(x)y = q(x),$$
\par где $p(x), q(x)$ - заданные на некотором интервале $I$ функции.
\subsection*{Метод Бернулли}
\par Работает для линейных неоднородных уравнений первого порядка и уравнений Бернулли.
\par Рассмотрим ДУ: $y' + p(x)y = q(x)$
\par Производим замену $y(x) = u(x) \cdot v(x) = u \cdot v, y' = u'v + uv'$
\par Подставим:
\par $u'v + uv' + p(x)uv = q(x)$
\par $u'v + uv' = q(x) - p(x)uv$
\par Приравняем слагаемые и получим систему: $$\left\{\begin{array}{l}
u'v = -p(x)uv \\
uv' = q(x)
\end{array}\right. \implies
\left\{\begin{array}{l}
\frac{du}{dx} = -p(x)u \\
\frac{dv}{dx} = \frac{q(x)}{u}
\end{array}\right. \implies
\left\{\begin{array}{l}
\frac{du}{u} = -p(x)dx \\
\frac{dv}{dx} = \frac{q(x)}{u}
\end{array}\right.$$
\par Получим (для $u(x)$ не берём константу): $$\left\{\begin{array}{l}
u = e^{\int -p(x) \, dx}  \\
v = \int \frac{q(x)}{e^{\int -p(x) \, dx}} \, dx + C
\end{array}\right.$$
\par Таким образом, общее решение: $$y = u \cdot v = e^{\int -p(x) \, dx}\int \frac{q(x)}{e^{\int -p(x) \, dx}} \, dx + C e^{\int -p(x) \, dx}$$
\subsection*{Метод Лагранжа}
\par Рассмотрим соответствующее однородное уравнение $y' + p(x)y = 0$
\par Это уравнение с разделяющимися переменными:
\par $\frac{dy}{y} = -p(x)dx$
\par $\ln y = \int -p(x) \, dx + C$
\par $y = Ce^{\int -p(x) \, dx}$ - общее решение однородного ДУ
\par Будем искать общее решение неоднородного ДУ в виде $y = C(x)e^{\int -p(x) \, dx}$, где $C(x)$ - неизвестная функция
\par Найдём производную: $y' = C'(x)e^{\int -p(x) \, dx} - p(x)C(x)e^{\int -p(x) \, dx}$
\par Подставим в исходное ДУ и найдём $C(x)$: $$C'(x)e^{\int -p(x) \, dx} - p(x)C(x)e^{\int -p(x) \, dx} + p(x)C(x)e^{\int -p(x) \, dx}y = q(x)$$
\par $$\frac{dC}{dx}e^{\int -p(x) \, dx} = q(x)$$
\par $$C(x) = \int q(x)e^{\int p(x) \, dx } \, dx + C^*$$
\par Подставим и получим общее решение:$$y = e^{\int -p(x)dx}\int \frac{q(x)}{e^{\int -p(x) \, dx}} \, dx + C^*e^{\int -p(x)dx}$$

\section*{22. Сформулировать теорему Коши о существовании и единственности решения дифференциального уравнения $n$-ого порядка. Интегрирование дифференциальных уравнений $n$-ого порядка, допускающих понижение порядка.}
\par\textbf{Теорема.} (Коши о $\exists$ и $!$ решения задачи Коши)
\par Пусть дано ДУ $y^{(n)} = f(x, y, ', \dots, y^{(n-1)})$, где функция $f$ определена в некоторой области $D$, содержащей точку $(x_{0}, y_{0}, y_{0}', \dots, y_{0}^{(n-1)})$. Пусть функция $f$ удовлетворяет следующим условиям:
\begin{itemize}
    \item $f(x, y, y', \dots, y^{(n-1)})$ - непрерывная функция $n+1$ переменной в области $D$
    \item все частные производные по переменным $y, y', \dots, y^{(n-1)}$ непрерывны и ограничены в этой области
\end{itemize}
\par Тогда найдётся интервал $(x_{0}-h, x_{0}+h)$, на котором $\exists$ и $!$ решение $y = \varphi(x)$ данного ДУ, удовлетворяющее начальным условиям $y(x_{0}) = y_{0}$, $y'(x_{0}) = y_{0}', \dots, y^{(n-1)}(x_{0}) = y_{0}^{(n-1)}$
\subsection*{Интегрирование ДУ $n$-ого порядка, допускающих понижение степени}
\begin{enumerate}
\item Решение уравнения вида $y^{(n)} = f(x)$ находится последовательным интегрированием $n$ раз:
\item[] $y^{(n-1)} = \int f(x) \, dx + C_{1}$
\item[] $y^{(n-2)} = \int \left(\int f(x) \, dx + C_{1} \, dx\right) + C_{2}$ и так далее
\item Если уравнение $F(x, y^{(k)}, y^{(k+1)}, \dots, y^{(n)})$ не содержит искомую функцию и её производные $y', \dots, y^{(k - 1)}$, порядок понижается на $k$ заменой: $y^{(k)} = p(x)$. Уравнение примет вид:
\item[] $F(x, p, p', \dots, p^{(n - k)}) = 0$
\item[] Находим его общее решение $p(x) = \psi(x, C_{1}, C_{2}, \dots, C_{n})$
\item[] Тогда $y^{(k)} = \psi(x, C_{1}, C_{2}, \dots, C_{n})$
\item[] Находим искомую функцию последовательным интегрированием $k$ раз и получаем общее решение ДУ.
\item Если уравнение $F(y, y', y'', \dots, y^{(n)})$ не содержит $x$, порядок ДУ можно понизить на 1 заменой $y' = p(y)$. Тогда $y'' = p(y)' = \frac{dp}{dy}y' = \frac{dp}{dy} p = p'p$ и так далее.
\item[] В случае ДУ 2-ого порядка получим $F(y, p, p'p) = 0$
\end{enumerate}

\section*{23. Сформулировать теорему Коши о существовании и единственности решения линейного дифференциального уравнения $n$-ого порядка. Доказать свойства частных решений линейного однородного дифференциального уравнения $n$-ого порядка.}
\par\textbf{Теорема.} (Коши о $\exists$ и $!$ решения задачи Коши)
\par Пусть дано линейное ДУ $y^{(n)} + a_{1}(x)y^{(n-1)} + a_{2}(x)y^{(n-2)} + \dots + a_{n}(x)y = g(x)$, где функция $g(x)$ и коэффициенты $a_{i}(x) \; (i = \overline{1, n})$ непрерывны в некоторой области $D$. Тогда для любой точки $(x_{0}, y_{0}, y_{0}', \dots, y_{0}^{(n-1)})$ существует и единственно решение $y = y(x)$ данного линейного ДУ, удовлетворяющее этим начальным условиям.
\subsection*{Свойства частных решений ЛОДУ $n$-ого порядка}
\par\textbf{Теорема 1.} Если функция $y_{0}(x)$ является решение ЛОДУ $L[y] = 0$, то функция $Cy_{0}(x)$, где  $C = const$, тоже является решением ЛОДУ $L[y] = 0$
\par\textbf{Доказательство.}
\par $y_{0}(x)$ - решение ЛОДУ $L[y] = 0$ по усл., $\implies$, $L[y_{0}] = 0$
\par Найдём (по свойству однородности): $L[Cy_{0}] = CL[y_{0}] = C \cdot 0 = 0$
\par $L[Cy_{0}] = 0 \implies Cy_{0}(x)$ является решение ЛОДУ $L[y] = 0 \Large_\blacktriangle$
\par\textbf{Теорема 2.} Если функции $y_{1}(x)$ и $y_{2}(x)$ являются решениями ЛОДУ $L[y] = 0$, то функция $y_{1}(x) + y_{2}(x)$ тоже является решение ЛОДУ $L[y] = 0$
\par\textbf{Доказательство.}
\par $y_{1}(x)$ и $y_{2}(x)$ - решения ЛОДУ $L[y] = 0$ по усл., $\implies$, $L[y_{1}] = 0, L[y_{2}] = 0$
\par Найдём (по свойству аддитивности): $L[y_{1} + y_{2}] = L[y_{1}] + L_{1}[y_{2}] = 0 + 0 = 0$
\par $L[y_{1} + y_{2}] = 0 \implies (y_{1}(x) + y_{2}(x))$ является решение ЛОДУ $L[y] = 0 \Large_\blacktriangle$
\par (далее необязательные свойства)
\par\textbf{Следствие.} Линейная комбинация с произвольными постоянными коэффициентами $C_{1}y_{1}(x) + C_{2}y_{2}(x) + \dots + C_{m}y_{m}(x)$ решений $y_{1}(x), y_{2}(x), \dots, y_{m}(x)$ ЛОДУ $L[y] = 0$ тоже является решением этого ЛОДУ.
\par\textbf{Доказательство.}
\par $L[y_{1}] = 0, L[y_{2}] = 0, \dots, L[y_{m}] = 0$ по условию
\par Найдём $L[C_{1}y_{1} + C_{2}y_{2} + \dots + C_{m}y_{m}] = L[C_{1}y_{1}] + L[C_{2}y_{2}] + \dots + L[C_{m}y_{m}] =$
\par $= C_{1}L[y_{1}] + C_{2}L[y_{2}] + \dots + C_{m}L[y_{m}] = 0$
\par $L[C_{1}y_{1} + C_{2}y_{2} + \dots + C_{m}y_{m}] = 0 \implies C_{1}y_{1}(x) + C_{2}y_{2}(x) + \dots + C_{m}y_{m}(x)$ является решением ЛОДУ $L[y] = 0 \Large_\blacktriangle$
\par\textbf{Утверждение.} ЛОДУ $L[y] = 0$ всегда имеет тривиальное решение $y \equiv 0$
\par\textbf{Теорема.} Совокупность решений ЛОДУ $L[y] = 0$ образует линейное пространство.


\section*{24. Сформулировать определения линейно зависимой и линейно независимой систем функций. Сформулировать и доказать теорему о вронскиане линейно зависимых функций.}
\par\textbf{Опр.} Функции $y_{1}(x), y_{2}(x), \dots, y_{n}(x)$ называются линейно-зависимыми на $[a, b]$, если существуют постоянные $\alpha_{1}, \alpha_{2}, \dots, \alpha_{n}$ такие, что на $[a, b]$ выполняется равенcтво $\alpha_{1}y_{1}(x) + \alpha_{2}y_{2}(x) + \dots + \alpha_{n}y_{n}(x) \equiv 0$, где хотя бы одна $\alpha_{i} \neq 0(i = 1, 2, \dots, n)$. Если же это тождество выполняется только при условии, что $\alpha_{1} = \alpha_{2} = \dots = \alpha_{n} = 0$, то функции $y_{1}(x), y_{2}(x), \dots, y_{n}(x)$ называются линейно-независимыми на $[a, b]$.
\par\textbf{Теорема.} Если функции $y_{1}(x), y_{2}(x), \dots, y_{n}(x)$ линейно зависимы на $[a, b]$, то $\forall x \in [a, b] \; W[y_{1}, y_{2}, \dots, y_{n}] = 0$
\par\textbf{Доказательство.}
\par По усл. $y_{1}(x), y_{2}(x), \dots, y_{n}(x)$ линейно зависимы на $[a, b]$, $\implies$, $\exists \alpha_{i} \neq 0$ такие, что $\alpha_{1}y_{1} + \alpha_{2}y_{2} + \dots + \alpha_{n}y_{n} = 0$. Дифференцируя $n-1$ раз получим систему:
$$\left\{\begin{array}{l}
\alpha_{1}y_{1} + \alpha_{2}y_{2} + \dots + \alpha_{n}y_{n} = 0 \\
\alpha_{1}y_{1}' + \alpha_{2}y_{2}' + \dots + \alpha_{n}y_{n}' = 0 \\
\dots \\
\alpha_{1}y_{1}^{(n-1)} + \alpha_{2}y_{2}^{(n-1)} + \dots + \alpha_{n}y_{n}^{(n-1)} = 0
\end{array}\right.$$
\par Получили СЛАУ с $n$ неизвестными $\alpha_{1}, \alpha_{2}, \dots, \alpha_{n}$
\par Так как хотя бы одна $\alpha_{i} \neq 0$, то эта система имеет ненулевое решение. Определителем такой системы является определитель Вронского $W[y_{1}, y_{2}, \dots, y_{n}]$. Полученная система имеет ненулевое решение лишь в том случаем, когда её определитель равен 0. То есть:
$$W(x) = \begin{vmatrix}
y_{1}(x) & y_{2}(x) & \dots & y_{n}(x) \\
y_{1}'(x) & y_{2}'(x) & \dots & y_{n}'(x) \\
\vdots & \vdots & & \vdots \\
y_{1}^{(n-1)}(x) & y_{2}^{(n-1)}(x) & \dots & y_{n}^{(n-1)}(x)
\end{vmatrix} = 0 \quad \forall x \in [a, b] _\blacktriangle$$

\section*{25. Сформулировать определения линейно зависимой и линейно независимой систем функций. Сформулировать и доказать теорему о вронскиане системы линейно независимых частных решений линейного однородного дифференциального уравнения $n$-ого порядка.}
\par\textbf{Опр.} Функции $y_{1}(x), y_{2}(x), \dots, y_{n}(x)$ называются линейно-зависимыми на $[a, b]$, если существуют постоянные $\alpha_{1}, \alpha_{2}, \dots, \alpha_{n}$ такие, что на $[a, b]$ выполняется равенcтво $\alpha_{1}y_{1}(x) + \alpha_{2}y_{2}(x) + \dots + \alpha_{n}y_{n}(x) \equiv 0$, где хотя бы одна $\alpha_{i} \neq 0(i = 1, 2, \dots, n)$. Если же это тождество выполняется только при условии, что $\alpha_{1} = \alpha_{2} = \dots = \alpha_{n} = 0$, то функции $y_{1}(x), y_{2}(x), \dots, y_{n}(x)$ называются линейно-независимыми на $[a, b]$.
\par\textbf{Теорема.} Если линейно независимые на $[a, b]$ функции $y_{1}(x), y_{2}(x), \dots, y_{n}(x)$ являются решениями ЛОДУ с непрерывными на $[a, b]$ коэффициентами $p_{i}(x) \; (i = \overline{1, n})$, то определитель Вронского этих функций отличен от нуля $\forall x \in [a, b]$
\par\textbf{Доказательство.} (методом от противного)
\par Допустим, что для какой-то точки $x_{0} \in [a, b] \; W(x_{0}) = 0$
\par Составим СЛАУ относительно $\alpha_{1}, \alpha_{2}, \dots, \alpha_{n}$:
$$\left\{\begin{array}{l}
\alpha_{1}y_{1}(x_{0}) + \alpha_{2}y_{2}(x_{0}) + \dots + \alpha_{n}y_{n}(x_{0}) = 0 \\
\alpha_{1}y_{1}'(x_{0}) + \alpha_{2}y_{2}'(x_{0}) + \dots + \alpha_{n}y_{n}'(x_{0}) = 0 \\
\dots \\
\alpha_{1}y_{1}^{(n-1)}(x_{0}) + \alpha_{2}y_{2}^{(n-1)}(x_{0}) + \dots + \alpha_{n}y_{n}^{(n-1)}(x_{0}) = 0
\end{array}\right.$$
\par В силу допущения определитель этой системы $W(x_0) = 0, x_{0} \in [a, b]$, $\implies$, эта система имеет ненулевое решение, то есть хотя бы одно из $\alpha_{1}, \alpha_{2}, \dots, \alpha_{n}$ отлично от нуля
\par Рассмотрим $y = \alpha_{1}y_{1}(x) + \alpha_{2}y_{2}(x) + \dots + \alpha_{n}y_{n}(x)$, то есть линейную комбинацию частных решений. Следовательно, эта функция сама является решением того же ЛОДУ, удовлетворяющим начальному условию $y(x_{0}) = y_{0}, y'(x_{0}) = y_{0}', \dots, y^{(n-1)}(x_{0}) = y_{0}^{(n-1)} = 0$
\par Но этим же начальным условиям удовлетворяет и тривиальное решение $y = 0$
\par По теореме о единственности решения: $\alpha_{1}y_{1}(x) + \alpha_{2}y_{2}(x) + \dots + \alpha_{n}y_{n}(x) = 0$ на $[a, b]$ и $\exists\alpha_{i} \neq 0$
\par По определению линейной зависимости функций $y_{1}(x), y_{2}(x), \dots, y_{n}(x)$ - линейно зависимые функции. Но это противоречит условию теоремы. Следовательно, предположение неверно и $\not\exists x_{0} \in [a, b]$ такой, что $W(x_{0}) \neq 0$.
\par То есть $W(x) \neq 0 \; \forall x \in [a, b] _{\blacktriangle}$

\section*{26. Сформулировать и доказать теорему о существовании фундаментальной системы решений линейного однородного дифференциального уравнения $n$-ого порядка.}
\par\textbf{Теорема.} У каждого ЛОДУ $n$-ого порядка $y^{(n)} + p_{1}(x)y^{(n-1)} + \dots + p_{n}(x)y = 0$ с непрерывными коэффициентами $p_{i}(x), i = \overline{1, n}$, существует ФСР
\par\textbf{Доказательство.}
\par Для построения ФСР зададим $n^2$ чисел (начальные условия):
\par \[\begin{array}{c c c c}
y_{1}(x_{0}) = y_{1_{0}} & y_{1}'(x_{0}) = y_{1_{0}}' & \dots & y_{1}^{(n-1)}(x_{0}) = y_{1_{0}}^{(n-1)} \\
y_{2}(x_{0}) = y_{2_{0}} & y_{2}'(x_{0}) = y_{2_{0}}' & \dots & y_{2}^{(n-1)}(x_{0}) = y_{2_{0}}^{(n-1)} \\
\dots & \dots & & \dots \\
y_{n}(x_{0}) = y_{n_{0}} & y_{n}'(x_{0}) = y_{n_{0}}' & \dots & y_{n}^{(n-1)}(x_{0}) = y_{n_{0}}^{(n-1)} 
\end{array}\]
\par Эти числа должны удовлетворять следующему условию:$$\begin{vmatrix}{}
y_{1_{0}} & y_{2_{0}} & \dots & y_{n_{0}} \\
y_{1_{0}}' & y_{2_{0}}' & \dots & y_{n_{0}}' \\
\dots & \dots & & \dots \\
y_{1_{0}}^{(n-1)} & y_{2_{0}}^{(n-1)} & \dots & y_{n_{0}}^{(n-1)}
\end{vmatrix} \neq 0$$
\par Точка $x_{0}$ - произвольная точка $\in [a, b]$
\par Тогда получается, что решение $y_{i}(x), i = \overline{1, n}$, удовлетворяет этим начальным условиям с определителем Вронского $W(x_{0}) \neq 0$. Следовательно, функции $y_{1}(x), y_{2}(x), \dots, y_{n}(x)$ линейно независимы на $[a, b]$ и образуют одну из ФСР ЛОДУ $n$-ого порядка.$_{\blacktriangle}$

\section*{27. Сформулировать и доказать теорему о структуре общего решения линейного однородного дифференциального уравнения $n$-ого порядка.}
\par\textbf{Теорема.} Общее решение на $[a, b]$ ЛОДУ $n$-ого порядка $L[y] = 0$ с непрерывными на $[a, b]$ коэффициентами $p_{i}(x) \;(i = \overline{1, n})$ равно линейной комбинации ФСР с произвольными постоянными коэффициентами, т.е.: $y_{\text{о.о.}} = C_{1}y_{1}(x) + C_{2}y_{2}(x) + \dots + C_{n}y_{n}(x)$, где $y_{1}(x), y_{2}(x), \dots, y_{n}(x)$ - ФСР ЛОДУ $L[y] = 0$, а $C_{1}, C_{2}, \dots, C_{n} - const$
\par\textbf{Доказательство.}
\par\textbf{1)} Докажем, что $C_{1}y_{1} + C_{2}y_{2} + \dots + C_{n}y_{n}$ - решение ЛОДУ $L[y] = 0$
\par Подставим его в ДУ:
\par $L[y] = L[C_{1}y_{1} + C_{2}y_{2} + \dots + C_{n}y_{n}] = C_{1}L[y_{1}] + C_{2}L[y_{2}] + \dots + C_{n}L[y_{n}] = 0$
\par Следовательно, $y = C_{1}y_{1} + C_{2}y_{2} + \dots + C_{n}y_{n}$ является решением ЛОДУ $L[y] = 0$
\par\textbf{2)} Докажем, что $y = C_{1}y_{1} + C_{2}y_{2} + \dots + C_{n}y_{n}$ - общее решение ЛОДУ $L[y] = 0$
\par По условию все коэффициенты есть непрерывные функции на  $[a, b]$, $\implies$, выполнены все условия теоремы Коши $\exists$ и $!$ решения ЛОДУ $L[y] = 0$.
\par Решение $y = C_{1}y_{1} + C_{2}y_{2} + \dots + C_{n}y_{n}$ будет общим решением, если найдутся единственным образом постоянные $C_{i}$ при произвольно заданных начальных условиях $y(x_{0}) = y_{0}$, $y'(x_{0}) = y_{0}', \dots, y^{(n-1)}(x_{0}) = y_{0}^{(n-1)}$, где $x_{0} \in [a, b]$
\par Пусть решение и его производные удовлетворяют этим условиям:
$$\left\{\begin{array}{l}
C_{1}y_{1}(x_{0}) + C_{2}y_{2}(x_{0}) + \dots + C_{n}y_{n}(x_{0}) = y_{0} \\
C_{1}y_{1}'(x_{0}) + C_{2}y_{2}'(x_{0}) + \dots + C_{n}y_{n}'(x_{0}) = y_{0}' \\
\dots \\
C_{1}y_{1}^{(n-1)}(x_{0}) + C_{2}y_{2}^{(n-1)}(x_{0}) + \dots + C_{n}y_{n}^{(n-1)}(x_{0}) = y_{0}^{(n-1)}
\end{array}\right.$$
\par Это неоднородная СЛАУ относительно $C_{1}, C_{2}, \dots, C_{n}$. Определитель этой системы является определителем Вронского $W(x_{0})$ для линейно независимой системы функций $y_{1}, y_{2}, \dots, y_{n}$ (решение ЛОДУ $L[y] = 0$) и тогда $W(x) \neq 0$. Следовательно, система имеет единственное решение $C_{1}, C_{2}, \dots, C_{n}$ для произвольной точки $(x_{0}, y_{0}, y_{0}', \dots, y_{0}^{(n-1)}) \implies$
\par $\implies y = C_{1}y_{1} + C_{2}y_{2} + \dots + C_{n}y_{n}$ - общее решение ЛОДУ $L[y] = 0 _{\blacktriangle}$

\section*{28. Вывести формулу Остроградского-Лиувилля для линейного дифференциального уравнения второго порядка.}
\par Рассмотрим ЛОДУ 2-ого порядка $y'' + p_{1}(x)y' + p_{2}(x)y = 0$, $p_{i}(x)$ - непрерывная на $[a, b]$ функция для $i = \overline{1, n}$.
\par Пусть $y_{1}(x), y_{2}(x)$ - решения этого ЛОДУ, тогда по определению:
$$\left\{\begin{array}{l}
y_{1}'' + p_{1}(x)y_{1}' + p_{2}(x)y_{1} = 0 \quad | \cdot (-y_{2})\\
y_{2}'' + p_{1}(x)y_{2}' + p_{2}(x)y_{2} = 0 \quad | \cdot y_{1}
\end{array}\right.$$
$$+\left\{\begin{array}{l}
- y_{1}''y_{2} - p_{1}(x)y_{1}'y_{2} - p_{2}(x)y_{1}y_{2} = 0 \\
y_{2}''y_{1} + p_{1}(x)y_{2}'y_{1} + p_{2}(x)y_{2}y_{1} = 0
\end{array}\right.$$
\par Сложив уравнения, получим:$$y_{2}''y_{1} - y_{1}''y_{2} + p_{1}(x)(y_{2}'y_{1} - y_{1}'y_{2}) = 0 \qquad (*)$$
\par Заметим, что:$$W(x) = \begin{vmatrix}{} y_{1} & y_{2} \\ y_{1}' & y_{2}' \end{vmatrix} = y_{1}y_{2}' - y_{2}y_{1}'$$
\par Тогда уравнение $(*)$ примет вид:$$y_{2}''y_{1} - y_{1}''y_{2} + p_{1}(x)W(x) = 0 \qquad (**)$$
\par Найдём:$$\frac{dW(x)}{dx} = (y_{1}y_{2}' - y_{1}'y_{2})' = y_{1}'y_{2}' + y_{1}y_{2}'' - y_{1}''y_{2} - y_{1}'y_{2}' = y_{1}y_{2}'' - y_{1}''y_{2}$$
\par Подставляя в $(**)$, получим:$$\frac{dW(x)}{dx} + p_{1}(x)W(x) = 0$$
$$\frac{dW(x)}{W(x)} = -p_{1}(x)dx$$
$$\int_{x_{0}}^x \frac{dW(x)}{W(x)} = -\int_{x_{0}}^x p_{1}(x) \, dx$$
$$\ln|W(x)| - \ln|W(x_{0})| = -\int_{x_{0}}^x p_{1}(x) \, dx$$
\par Тогда получим формулу Остроградского-Лиувилля:$$ W(x) = W(x_{0}) e^{-\int_{x_{0}}^x p_{1}(x) \, dx }$$

\section*{29. Вывести формулу для общего решения линейного однородного дифференциального уравнения второго порядка при одном известном частном решении.}
\par\textbf{Нахождение общего решения ЛОДУ 2-ого порядка при известном частном решении:}
\par Дано ЛОДУ $y''' + p_{1}(x)y' + p_{2}y = 0$ и $y_{1}$ - его известное частное решение.
\par Найдём второе решение этого ДУ, которое будет линейно независимо с $y_{1}$:
$$W(x) = \begin{vmatrix}
y_{1} & y_{2}  \\
y_{1}' & y_{2}'
\end{vmatrix} = y_{1}y_{2}' - y_{1}'y_{2}, W(x) \neq 0$$
\par Найдём производную:
$$\left( \frac{y_{2}}{y_{1}} \right)' = \frac{y_{1}y_{2}' - y_{1}'y_{2}}{y_{1}^2} = \frac{W(x)}{y_{1}^2}$$
\par Используя формулу Остроградского-Лиувилля, получим:
$$\frac{W(x)}{y_{1}^2} = \frac{e^{-\int p_{1}(x) \, dx }}{y_{1}^2} \implies \frac{y_{2}}{y_{1}} = \int \frac{e^{-\int p_{1}(x) \, dx }}{y_{1}^2} \, dx$$
\par Отсюда:$$y_{2} = y_{1} \int \frac{e^{-\int p_{1}(x) \, dx }}{y_{1}^2} \, dx$$
\par Докажем, что найденное решение $y_{2}$ линейно независимо с данным решением $y_{1}$. Найдём $W[y_{1}, y_{2}]$:
$$W(x) = \begin{vmatrix}
y_{1} & y_{2}  \\
y_{1}' & y_{2}'
\end{vmatrix} = 
\begin{vmatrix}
y_{1} & y_{1} \int \frac{e^{-\int p_{1}(x) \, dx }}{y_{1}^2} \, dx   \\
y_{1}' & y_{1}'\int \frac{e^{-\int p_{1}(x) \, dx }}{y_{1}^2} \, dx + \frac{e^{-\int p_{1}(x) \, dx }}{y_{1}} 
\end{vmatrix} =$$
$$= y_{1}y_{1}'\int \frac{e^{-\int p_{1}(x) \, dx }}{y_{1}^2} \, dx + e^{\int p_{1}(x) \, dx } - y_{1}y_{1}'\int \frac{e^{-\int p_{1}(x) \, dx }}{y_{1}^2} \, dx = e^{-\int p_{1}(x) \, dx} > 0 \; \forall x$$
\par Следовательно, функции $y_{1}$ и $y_{2}$ образуют ФСР. Общее решение имеет вид:
$$y=C_{1}y_{1} + C_{2}y_{1}\int \frac{e^{-\int p_{1}(x) \, dx }}{y_{1}^2} \, dx $$


\section*{30. Сформулировать и доказать теорему о структуре общего решения линейного неоднородного дифференциального уравнения $n$-ого порядка.}
\par\textbf{Теорема.} Общее решение ЛНДУ $n$-ого порядка с непрерывными на $[a, b]$ коэффициентами $p_{i}(x), i = \overline{1, n}$ и функцией $f(x)$ (правая часть) равно сумме общего решения однородного ДУ и какого-либо частного решения неоднородного ДУ: $y_{\text{о.н.}} = y_{\text{о.н.}} + y_{\text{ч.н.}}$
\par\textbf{Доказательство.}
\par\textbf{1)} Докажем, что $y_{\text{о.н.}}$ есть решение ДУ. По условию $L[y_{\text{ч.н.}}] = f(x)$, $L[y_{\text{о.о.}}] = 0$
\par $L[y_{\text{о.н}}] = L[y_{\text{о.о.}} + y_{\text{ч.н.}}] = L[y_{\text{о.о.}}] + L[y_{\text{ч.н.}}] = 0 + f(x) = f(x)$
\par Следовательно, $y_{\text{о.н.}}$ - решение ДУ
\par\textbf{2)} Докажем, что $y_{\text{о.н.}} = y_{\text{о.о.}} + y_{\text{ч.н.}}$ - общее решение
\par $y_{\text{о.н.}} = y_{\text{о.о.}} + y_{\text{ч.н.}} = \sum_{i=1}^n C_{i} y_{i} + y_{\text{ч.н.}} =$ (по теореме о структуре общего решения)
\par $= C_{1}y_{1} + C_{2}y_{2} + \dots + C_{n} y_{n} + y_{\text{ч.н.}}$, где $y_{1}, y_{2}, \dots, y_{n}$ - линейно независимые частные решения соответствующего ЛОДУ, причём:
$$W(X) = \begin{vmatrix}{} \\
y_{1} & y_{2} & \dots & y_{n} \\
y_{1}' & y_{2}' & \dots & y_{n}' \\
\dots & \dots & \dots & \dots \\
y_{1}^{(n-1)} & y_{2}^{(n-1)} & \dots & y_{n}^{(n-1)} \\
\end{vmatrix} \neq 0 \quad \forall x \in [a, b]$$
\par Так как коэффициенты $p_i(x), i = \overline{1, n}$, - непрерывны на $[a, b]$, то по теореме Коши о существовании и единственности решения задачи Коши существует единственное решение ДУ, удовлетворяющее заданным условиям. Следовательно, надо доказать, что если решение $y_{\text{о.н.}} = C_{1}y_{1} + C_{2}y_{2} + \dots + C_{n} y_{n} + y_{\text{ч.н.}}$ и его производные удовлетворяют заданным начальным условиям, то из этих условий можно единственным образом определить $C_{1}, C_{2}, \dots, C_{n}, x_{0} \in [a, b]$
$$\left\{\begin{array}{l}
C_{1}y_{1}(x_{0}) + C_{2}y_{2}(x_{0}) + \dots + C_{n}y_{n}(x_{0}) = y_{0} - y_{\text{ч.н.}}(x_{0}) \\
C_{1}y_{1}'(x_{0}) + C_{2}y_{2}'(x_{0}) + \dots + C_{n}y_{n}'(x_{0}) = y_{0}' - y_{\text{ч.н.}}'(x_{0}) \\
\dots \\
C_{1}y_{1}^{(n-1)}(x_{0}) + C_{2}y_{2}^{(n-1)}(x_{0}) + \dots + C_{n}y_{n}^{(n-1)}(x_{0}) = y_{0}^{(n-1)} - y_{\text{ч.н.}}^{(n-1)}(x_{0})
\end{array}\right.$$
\par СЛАУ с определителем $W(x) \neq 0$, $x_{0} \in [a, b]$, $\implies$, существует единственный набор $C_{1} = C_{1}^0, C_{2} = C_{2}^0, \dots, C_{n}^0$
$y(x) = C_{1}^0y_{1}(x) + C_{2}^0y_{2}(x) + \dots + C_{n}^0y_{n}(x) + y_{\text{ч.н.}}$ - частное решение
\par Итак, $y_{\text{о.н.}} = y_{\text{о.о.}} + y_{\text{ч.н}}$ $_{\blacktriangle}$

\section*{31. Вывести формулу для общего решения линейного неоднородного дифференциального уравнения второго порядка с постоянными коэффициентами в случае кратных корней характеристического уравнения.}
\par Дано ЛОДУ 2-ого порядка: $y'' + a_{1}y' + a_{2}y = 0$, $a_{1}, a_{2} - const$
\par Будем искать решение в виде $y = e^{\lambda x}$
\par Найдём $y' = \lambda e^{\lambda x}$ и $y'' = \lambda^2 e^{\lambda x}$
\par Подставим в исходное ДУ, после упрощения получим характеристическое уравнение:
\par $\lambda^2 + a_{1} \lambda + a_{2} = 0$
\par $D = a_{1}^2 - 4a_{2}$
\par $\lambda_{1} = \frac{-a_{1} + \sqrt{ D }}{2}$, $\lambda_{2} = \frac{-a_{1} - \sqrt{ D }}{2}$
\par Пусть $D = 0$: $\lambda_{1} = \lambda_{2}$ - действительные корни
\par $\lambda = \lambda_{1} = \lambda_{2} = -\frac{a_{1}}{2}$
\par $a_{1} = -2 \lambda$
\par Первое частное решение: $y = e^{\lambda x}$
\par Найдём второе частное решение:
$$y_{2} = y_{1} \int \frac{e^{-\int a_{1} \, dx}}{y_{1}^2} \, dx = e^{\lambda x} \int \frac{e^{- a_{1} x}}{e^{2\lambda x}} \, dx = e^{\lambda x} \int \frac{e^{2\lambda x}}{e^{2\lambda x}} \, dx = x e^{\lambda x}$$
\par ФСР: $y_{1} = e^{\lambda x}, y_{2} = x e^{\lambda x}$
\par $y_{\text{о.о.}} = C_{1}y_{1} + C_{2}y_{2} = C_{1} e^{\lambda x} + C_{2} x e^{\lambda x}$

\section*{32. Вывести формулу для общего решения линейного неоднородного дифференциального уравнения второго порядка с постоянными коэффициентами в случае комплексных корней характеристического уравнения.}
\par Дано ЛОДУ 2-ого порядка: $y'' + a_{1}y' + a_{2}y = 0$, $a_{1}, a_{2} - const$
\par Будем искать решение в виде $y = e^{\lambda x}$
\par Найдём $y' = \lambda e^{\lambda x}$ и $y'' = \lambda^2 e^{\lambda x}$
\par Подставим в исходное ДУ, после упрощения получим характеристическое уравнение:
\par $\lambda^2 + a_{1} \lambda + a_{2} = 0$
\par $D = a_{1}^2 - 4a_{2}$
\par $\lambda_{1} = \frac{-a_{1} + \sqrt{ D }}{2}$, $\lambda_{2} = \frac{-a_{1} - \sqrt{ D }}{2}$
\par Пусть $D < 0$: $\lambda_{1}, \lambda_{2}$ - комплексно сопряжённые
\par $\lambda_{1, 2} = \alpha \pm \beta i, (\beta \neq 0)$
\par Рассмотрим $e^{\lambda_{1}x} = e^{(\alpha + \beta i)x} = e^{\alpha x}(\cos \beta x + i\sin \beta x)$ - формула Эйлера
\par Выделим действительную и мнимую части решения:
\par $y_{1} = e^{\alpha x} \cos \beta x$ и $y_{2} = e^{\alpha x} \sin \beta x$
$$W(x) = \begin{vmatrix}
e^{\alpha x} \cos \beta x & e^{\alpha x} \sin \beta x \\
\alpha e^{\alpha x} \cos \beta x - e^{\alpha x} \beta \sin \beta x & \alpha e^{\alpha x} \sin \beta x + e^{\alpha x} \beta \cos \beta x
\end{vmatrix} =$$ 
$$= \alpha e^{2\alpha x} \sin \beta x \cos \beta x + e^{2\alpha x} \beta \cos^2 \beta x - \alpha e^{2\alpha x} \sin \beta x \cos \beta x + e^{2\alpha x} \beta \sin^2 \beta x =$$
$$= e^{2\alpha x}\neq 0 \; \forall x \in [a, b]$$
\par То есть $y_{1} = e^{\alpha x} \cos \beta x$ и $y_{2} = e^{\alpha x} \sin \beta x$ линейно независимые частные решения ДУ и образуют ФСР, по теореме о структуре общего решения ЛОДУ:
\par $y_{\text{о.о.}} = C_{1}y_{1} + C_{2}y_{2} = C_{1}e^{\alpha x} \cos \beta x + C_{2}e^{\alpha x} \sin \beta x$

\section*{33. Частное решение линейного неоднородного дифференциального уравнения с постоянными коэффициентами и правой частью специального вида (являющейся квазимногочленом). Сформулировать и доказать теорему о наложении частных решений.}
\par Дано ЛНДУ $y^{(n)} + a_{1}y^{(n-1)} + \dots + a_{n-1}y' + a_{n}y = f(x)$, где $a_{i} = const \; (i = \overline{1, n})$.
\par Правая часть имеет вид: $e^{\alpha x} [P_{n}(x)\cos\beta x + Q_{m}(x)\sin\beta x]$, где $P_{n}(x)$ и $Q_{m}(x)$ - многочлены от $x$ степеней $n$ и $m$ соответственно, $\alpha, \beta \in \mathbb{R}$
\par Рассмотрим соответствующее однородное ДУ $y^{(n)} + a_{1}y^{(n-1)} + \dots + a_{n-1}y' + a_{n}y = 0$
\par Характеристическое уравнение: $\lambda^{n} + a_{1}\lambda^{(n-1)} + \dots + a_{n-1}\lambda + a_{n} = 0$
\par $y_{\text{о.н.}} = y_{\text{о.о.}} +y_{\text{ч.н.}}$
\par Частное решение неоднородного ДУ находим в виде:
\par $$y_{\text{ч.н.}} = x^{r}e^{\alpha x}[R_{s}(x)\cos\beta x + T_{s}(x)\sin\beta x],$$
\par $r$ - кратность корней $\alpha \pm \beta i$ в характеристическом уравнении, $r = 0$, если корнем не является
\par $s = \max(n, m), R_{s}(x), T_{s}(x)$ - общий вид многочленов степени $s$
\par Неопределённые коэффициенты находим, подставляя решение в исходное ДУ.
\par Соответствие между видом правой части неоднородного ДУ и видом его частного решения

\subsection*{Доказательство теоремы}
\par\textbf{Теорема.} Если $y_{1}(x)$ есть решение уравнения $L[y] = f_{1}(x)$, а $y_{2}(x)$ есть решение уравнения $L[y] = f_{2}(x)$, то функция $y_{1}(x) + y_{2}(x)$ есть решение уравнения $L[y] = f_{1}(x) + f_{2}(x)$
*\par\textbf{Доказательство.}
\par По условию $L[y_1] = f_{1}(x)$, $L[y_2] = f_{2}(x)$
\par Найдём $L[y_{1} + y_{2}] = L[y_{1}] + L[y_{2}] = f_{1}(x) + f_{2}(x)$
\par Следовательно, функция $y_{1}(x) + y_{2}(x)$ есть решение уравнения $L[y] = f_{1}(x) + f_{2}(x)\Large_{\blacktriangle}$

\section*{34. Метод Лагранжа вариации произвольных постоянных для нахождения решения линейного неоднородного дифференциального уравнения 2-ого порядка и вывод системы соотношений для варьируемых переменных.}
\par Дано ЛНДУ $y'' + p_{1}(x)y' + p_{2}(x)y = f(x)$ с непрерывными коэффициентами $p_{i}(x), i = \overline{1, n}$
\par Пусть $y_{1}(x), y_{2}(x)$ - ФСР соответствующего ЛОДУ
\par Будем искать решение ЛНДУ в виде: $y = C_{1}(x)y_{1}(x) + C_{2}(x)y_{2}(x) = C_{1}y_{1} + C_{2}y_{2}$
\par где $C_{1}(x)$, $C_{2}(x)$ - новые неизвестные функции
\par Найдём $y' = C_{1}'y_{1} + C_{1}y_{1}' + C_{2}'y_{2} + C_{2}y_{2}'$
\par Наложим ограничение: $C_{1}'y_{1} + C_{2}'y_{2} = 0$
\par Тогда $y' = C_{1}y_{1}' + C_{2}y_{2}'$
\par Найдём $y'' = C_{1}'y_{1}' + C_{1}y_{1}'' + C_{2}'y_{2}' + C_{2}y_{2}''$
\par Подставим найденные $y, y', y''$ в исходное ЛНДУ и упростим:
\par $C_{1}'y_{1}' + C_{2}'y_{2}' + C_{1}(y_{1}'' + p_{1}y_{1}' + p_{2}y_{2}) + C_{2}(y_{2}'' + p_{1}y_{2}' + p_{2}y_{2}) = f(x)$
\par $y_{1}'' + p_{1}y_{1}' + p_{2}y_{1} = 0$ и $y_{2}'' + p_{1}y_{2}' + p_{2}y_{2} = 0$ по условию
\par Получим $C_{1}'y_{1}' + C_{2}'y_{2} = f(x)$
\par Это верно только при наложенном ограничении, то есть:$$\left\{\begin{array}{l}
C_{1}'(x)y_{1} + C_{2}'(x)y_{2} = 0 \\
C_{1}'(x)y_{1}' + C_{2}'(x)y_{2}' = f(x) \\
\end{array}\right.$$
\par Определителем этой системы является определитель Вронского $W(x) = \begin{vmatrix}{} y_{1} & y_{2} \\ y_{1}' & y_{2}' \end{vmatrix} \neq 0$, так как $y_{1}, y_{2}$ составляют ФСР исходного ДУ. Следовательно, коэффициенты $C_{1}, C_{2}$ определены единственным образом.
\par Пусть $C_{1}' = \varphi_{1}(x), C_{2}' = \varphi_{2}(x)$
\par Тогда $C_{1} = \int \varphi_{1}(x) \, dx, C_{2} = \int \varphi_{2} (x)\, dx$
\par Общее решение ЛНДУ: $y_{\text{о.о.}} = C_{1}y_{1} + C_{2}y_{2} + y_{1}\int \varphi_{1}(x) \, dx + y_{2} \int \varphi_{2}(x) \, dx$

\end{document}