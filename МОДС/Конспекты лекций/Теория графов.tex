\documentclass[10pt]{article}
% for russian lang
\usepackage[T2A]{fontenc}
\usepackage[english, russian]{babel}
% end for russian lang

%\nonstopmode

% config page
%\usepackage[paperheight=6in,
%   paperwidth=5in,
%   top=10mm,
%   bottom=20mm,
%   left=10mm,
%   right=10mm]{geometry}

\usepackage[a4paper, margin=20mm]{geometry}
\usepackage{amsfonts}
\usepackage{amsmath}
\usepackage{amssymb}
\usepackage{indentfirst}
\usepackage{csquotes}
\usepackage{graphicx}
\graphicspath{ {./images/} }

\title{Теория графов}
\author{ИУ6-25Б}
\date{2024}

\begin{document}
\maketitle

%%%%%%%%%%%%%%%%%%%%%%%%%%%%%%% 1 file

\section*{Задача о кёнингсбергских мостах}
\begin{center}
    \includegraphics[width=\textwidth]{bridges}
\end{center}
\par\textbf{Опр.} Графом $G(X, U)$ называется математический объект, заданный парой множеств $X = {x_{1}, x_{2}, \dots, x_{n}}$ - множество вершин - и $U = {i_{1}, u_{2}, \dots, u_{m}}$ - множество рёбер.
\par Так как любые 2 вершины могут быть связаны или не связаны ребром, то на множестве вершин задаётся бинарное отношение.
\par\textbf{Опр.} Ребро, заданное неупорядоченной парой $u_{l} = \{ x_{i}, x_{j} \}$ называется неориентированным.
\par\textbf{Опр.} Граф, состоящий только из неориентированных рёбер, называет неориентированным, неографом.
\par\textbf{Опр.} Ребро, заданное упорядоченной парой вершин $u_{l} = (x_{i}, x_{j})$, называется ориентированным.
\par\textbf{Опр.} Граф, состоящий только из ориентированных рёбер, называет ориентированным, орграфом.
\par\textbf{Опр.} Граф, состоящий из ориентированных и неориентированных рёбер, называется смешанным.
\par\textbf{Опр.} Если в графе хотя бы одну пару вершин связывает несколько рёбер, то такой граф называется мультиграфом.
\par\textbf{Опр.} Вершины, определяющие ребро, называются концевыми вершинами.
\par\textbf{Опр.} Если концевые вершины ребра совпадают, то такое ребро называется петлей.
\subsection*{Понятие смежности и инцидентности}
\par Между вершинами и рёбрами графа имеет место отношение инцидентности: вершина инцидентна ребру, если является одной из его концевых вершин.
\par Между вершинами или между рёбрами имеет место отношение смежности:
\begin{itemize}
    \item Два ребра называются смежными, если имеют общую концевую вершину.
    \item Две вершины называются смежными, если являются концевыми вершинами одного ребра.
\end{itemize}
\par Отношение инцидентности имеет место быть между однородными компонентами графа, отношение смежности - между однородными компонентами.
\par Отношение смежности является отношением толерантности, если принять, что вершина(ребро) смежна сама себе.

\par\textbf{Опр.} Граф называется простым, если в нём нет петель и кратных рёбер.
\par\textbf{Опр.} Граф называется полным, если 2 любые его вершины смежны.
\par Полный граф обязан быть простым.

\par\textbf{Опр.} Граф называется графом Кёнига (двудольным графом), если множество вершин можно разбить на 2 непересекающихся множества $X_{1}$ и $X_{2}$ ($X_{1} \cap X_{2} = \varnothing$) таких, что для каждого ребра верно, что одна из его вершин принадлежит $X_{1}$, а другая - $X_{2}$
\begin{itemize}
    \item $\forall u_{l} \in U, u_{l} = (x_{i}, x_{j}): x_{i} \in X_{1}, x_{j} \in X_{2}, i \not= j$
    \item $X_{1}$ и $X_{2}$ называются долями двудольного графа
\end{itemize}

\subsection*{Понятие смежности}
\par\textbf{Опр.} $|X| = n$ - мощность множества вершин называется порядком графа
\par\textbf{Опр.} Число рёбер, инцидентных вершине $x_{i} \in X$, называется степенью вершины $x_{i}$
\par\textbf{Обоз.} $p(x_{i}) = n$
\par\textbf{Опр.} Если $p(x_{i}) = 1$, то вершина $x_{i}$ называется висячей
\par\textbf{Опр.} Если $p(x_{i}) = 0$, то вершина $x_{i}$ называется изолированной
\par В случае орграфа:
\begin{itemize}
\item $p^+(x_{i})$ - полустепень захода - количество ориентированных рёбер, входящих в вершину $x_{i}$
\item $p^-(x_{i})$ - полустепень исхода - количество ориентированных рёбер, исходящих из вершины $x_i$
\item $p(x_{i}) = p^+(x_{i}) + p^-(x_{i})$
\end{itemize}

\par Для произвольного графа $G$ верно, что сумма степеней его вершин равна удвоенному числу его рёбер: $\Sigma p(x_{i}) = 2 |U| \; x_{i} \in X$ - лемма о рукопожатиях
\par Каждое ребро инцидентно двум вершинам, отсюда 2 в правой части. Каждое ребро принимает участие в формировании степени обеих своих концевых вершин.
\par\textbf{Опр.} Если $p^-(x_{i}) = 0$, то вершина $x_{i}$ называется стоком. ($x_{i} = t$)
\par\textbf{Опр.} Если $p^+(x_{i}) = 0$, то вершина $x_{i}$ называется источником. ($x_{i} = s$)
\par\textbf{Опр.} Граф с одним источником и одним стоком называется сетью

%%%%%%%%%%%%%%%%%%%%%%%%%%%%%%% 2 file
\section*{Изоморфизм графов}
\par\textbf{Опр.} Рассмотрим графы $G_{1} (X_{1}, U_{1})$ и $G_{2}(X_{2}, U_{2})$. Они называются изоморфными, если между множествами вершин $X_{1}$ и $X_{2}$ установлено взаимно однозначное соответствие такое, что между $U_{1}$ и $U_{2}$ устанавливается также взаимно однозначное соответствие. А именно каждое ребро из $U_{2}$ инцидентно 2 своим концевым вершинам из $X_{2}$ $\iff$ когда соответствующие им вершины из $X_{1}$ инцидентны ребру из множества $U_{1}$
\par\textbf{Теорема.} Изоморфизм графом является отношением эквивалентности:
\begin{enumerate}
    \item Рефлексивность - граф изоморфен сам себе $G \leftrightarrow G$
    \item Симметричность - $f: G_{1} \leftrightarrow G_{2} \implies f^{-1}: G_{2} \leftrightarrow G_{1}$
    \item Транзитивность:
    \begin{itemize}
        \item $f_{1}: G_{1} \leftrightarrow G_{2}$
        \item $f_{2}: G_{2} \leftrightarrow G_{3}$
        \item $f_3 = f_{1} \circ f_{2}$
        \item $f_{3}: G_{1} \leftrightarrow G_{3}$
    \end{itemize}
\end{enumerate}

\par Из теоремы следует, что множество всех графов разбивается на классы эквивалентности такие, что в пределах каждого класса все графы изоморфны, а графы из разных классов не изоморфны (*"с точностью до изоморфизма"*).
\subsection*{Части графа}
\par Рассмотрим граф $G(X, U)$
\par\textbf{Опр.} Граф $G_{1}(X_{1}, U_{1})$ называется частью графа $G$, если он находится в отношении включения к нему: $G_{1} \subseteq G \; (X_{1} \subseteq X, U_{1} \subseteq U)$
\par\textbf{Опр.} Часть графа называется подграфом, если включение строгое: $G_{1} \subset G \; (X_{1} \subset X, U_{1} \subset U)$
\par Обозначение удаления ребра: $G - (x_{i}, x_{j})$
\par\textbf{Опр.} Часть графа называется субграфом, если $X_{1} = X$, a $U_{1} \subset U$
\section*{Способы задания графа}
\begin{enumerate}
    \item Матричный
    \item Аналитический - основан на понятии отображения
        \begin{itemize}
            \item \par\textbf{Обоз.}  $\text{Г}_{x_{i}}$ - прямое отображение вершины $x_i$
            \item \textit{Пример:}
                \begin{itemize}
                    \item $\text{Г}_{x_{1}} = \{ x_{2} \}$ - исходящие рёбра
                    \item $\text{Г}_{x_{2}} = \varnothing$
                    \item $\text{Г}_{x_{3}} = \{ x_{2} \}$
                    \item $\text{Г}_{x_{4}} = \varnothing$
                    \item $\text{Г}_{x_{5}} = \{ x_{1} \}$
                    \item $\text{Г}_{x_{6}} = \varnothing$
                    \item $\text{Г}^{-1}_{x_{1}} = \{ x_{5} \}$ - входящие рёбра
                \end{itemize}
        \end{itemize}
    \item Списковые структуры данных. Вектор Айлифа:
        \begin{center}
            \includegraphics[width=0.3\paperwidth]{vector}
        \end{center}
    \item Массив рёбер или пар смежных вершин:
        \par\begin{tabular}{|c|c|}
            \hline
            $x_{1}$ & $x_{2}$ \\
            \hline
            $x_{5}$ & $x_{1}$ \\
            \hline
            $x_{1}$ & $x_{4}$ \\
            \hline
            $x_{3}$ & $x_{2}$ \\
            \hline
            $x_{5}$ & $x_{6}$ \\
            \hline
        \end{tabular}
\end{enumerate}
\subsection*{Матричный способ}
\subsubsection*{1. Матрица смежности}
\par Матрица смежности - квадратная булева матрица $M$ порядка $n = |X|$.
\begin{itemize}
    \item $M_{ij} = 1$, если $x_{i}$ и $x_{j}$ смежны
    \item $M_{ij} = 0$, если $x_{i}$ и $x_{j}$ не смежны
\end{itemize}
\par \begin{tabular}{|c|c|c|c|c|c|c|}
    \hline
    & $x_{1}$ & $x_{2}$ & $x_{3}$ & $x_{4}$ & $x_{5}$ & $x_{6}$ \\
    \hline
    $x_{1}$ & & $1$ & & $1$ & $1$ & \\
    \hline
    $x_{2}$ & $1$ & & $1$ & & & \\
    \hline
    $x_{3}$ & & $1$ & & & & \\
    \hline
    $x_{4}$ & $1$ & & & & & \\
    \hline
    $x_{5}$ & $1$ & & & & & $1$ \\
    \hline
    $x_{6}$ & & & & & $1$ & \\
    \hline
\end{tabular}
\begin{itemize}
    \item Сумма элементов равна числу рёбер графа
    \item Самый быстрый, но затратный по памяти способ
    \item Матрицу смежности можно построить для рёбер
\end{itemize}
\subsubsection*{2. Матрица инциденций}
\par Матрица порядка $m \times n$, где $m = |X|, n = |U|$
\par Для неографов:
\begin{itemize}
    \item $H_{ij} = 1$, если $x_{i}$ инцидентна $u_{j}$
    \item $H_{ij} = 0$, иначе
    \item Сумма столбца = 2
\end{itemize}
Для орграфов:
\begin{itemize}
    \item $H_{ij} = 1$, если $x_{i}$ инцидента $u_{j}$ и является конечной для него
    \item $H_{ij} = 01$, если $x_{i}$ не инцидента $u_{j}$
    \item $H_{ij} = -1$, если $x_{i}$ инцидента $u_{j}$ и является начальной для него
    \item Сумма столбца = 0
\end{itemize}

\par По строкам матрицы можно вычислить степени и полустепени вершин.

%%%%%%%%%%%%%%%%%%%%%%%%%%%%%%% 3 file

\par\textbf{Опр.} Маршрутом, соединяющим вершины $x_{i}, x_{j}$ в графе, называется чередующаяся последовательность $x_{i}u_{q}x_{w}u_{e}\dots u_{k}x_{j}$
\begin{itemize}
    \item Любые 2 соседних элемента связаны отношением инцидентности
    \item Любые 2 соседних через один связаны отношением смежности относительно стоящего между ними
    \item Маршрут не учитывает ориентацию рёбер
    \item Вершины и рёбра могут повторяться
\end{itemize}

\par\textbf{Опр.} Маршрут без повторяющихся рёбер называют цепью
\par\textbf{Опр.} Цепь, все вершины которой различны, называют простой
\par\textbf{Опр.} Простая цепь, начальная и конечная вершины которой совпадают, называется циклом
\par\textbf{Опр.} Цикл в орграфе называют контуром
\par\textbf{Опр.} Цепь в орграфе называют путём
\par\textbf{Опр.} Число рёбер, составляющих в цепь, называют длиной цепи или пути

\subsection*{Связь понятий маршрута и связности}
\par\textbf{Опр.} Граф является связным тогда и только тогда, когда любые 2 различные его вершины соединены маршрутом
\par\textbf{Опр.} Сильная связность в орграфе подразумевает, что между 2 любыми вершинами существует путь, слабая связность игнорирует направление рёбер
\par Отношение связности на множестве вершин графа является отношением эквивалентности:
\begin{itemize}
    \item Рефлексивно, так любая вершина связна сама с собой
    \item Симметрично, так как для любого маршрута существует обратный по тем же рёбрам
    \item Транзитивно, так как $x_{i} \to x_{j}, x_{j} \to x_{k} \implies x_{i} \to x_{k}$
\end{itemize}

\par Отношение связности разбивает множество вершин графа на классы эквивалентности.
\par\textbf{Опр.} Вершина графа $x_{i}$ называется точкой сочленения, если её удаление из графа увеличивает число компонент связности.
\par\textbf{Опр.} Если существует хотя бы одна точка сочленения, то граф называется разделимым (иначе неразделимым)
\par\textbf{Теорема.} Вершина $x_{k}$ связного графа $G(X, U)$ является точкой сочленения, если и только если в графе $\exists x_{i} \neq x_{j} \neq x_{k}$ такие, что любой путь или любая цепь между $x_{i}$ и $x_{j}$ проходит через $x_{k}$
\par\textbf{Опр.} Если в графе 1 точка сочленения, он называется односвязным.
\par\textbf{Опр.} Граф называют $i$-связным, если для нарушения его связности необходимо удалить не менее $i$ вершин. $i$ называют числом связности.

\section*{Обходы (Эйлеровы и Гамильтоновы)}
\par\textbf{Опр.} Эйлеровым циклом (обходом) в неографе называют цикл, включающий все рёбра графа и проходящий по каждому ребру только один раз
\par Можно выделить часть графа, в которой существует эйлеров обход.
\par\textbf{Опр.} Граф, содержащий эйлеров цикл, называют эйлеровым
\par\textbf{Теорема Эйлера.} Граф является эйлеровым тогда и только тогда, когда все его вершины имеют чётную степень - сформулирована для неографов
\par\textbf{Доказательство.}
\par\textbf{Необходимость} ($\Rightarrow$).
\par Пусть связные неограф $G$ эйлеров. Следовательно, цикл в $G$ проходит через каждую вершину. Для каждой вершины верно: цикл входит в вершину по одному ребру, а выходит по другому. Таким образом, каждая вершина инцидентна чётному числу рёбер. Так как эйлеров цикл, согласно определению, содержит все рёбра графа, то степени всех вершин чётны.
\par\textbf{Достаточность} ($\Leftarrow$).
\par Пусть степени всех вершин связного неографа $G$ чётны. Начнём построение цепи $C_{1}$ из вершины $x_{1}$. Попав в очередную вершину $x_{i}$ по некоторому ребру, всегда возможно выйти из неё по другому ребру, так как степень каждому вершины чётная. Тогда окончание цепи $C_{1}$ придётся на вершину $x_{1}$, то есть $C_{1}$ будет циклом. Если при получении цикла $C_{1}$ были пройдены все ребра графа $G$, то $G$ - эйлеров граф.
\par Если были пройдены не все рёбра графа, то будем считать, что пройденные рёбра помечены. Тогда для любой вершины, имеющей непомеченные рёбра среди инцидентных ей рёбер, верно: число непомеченных рёбер чётное. Найдём первую такую вершину $x_{i}$ и начнём от неё построение цепи $C_{2}$ по непомеченным рёбрам. В силу чётности степеней всех вершин и чётности числа непомеченных рёбер имеем: $C_{2}$ заканчивается в вершине $x_{i}$ и будет циклом. Так как граф связный, то циклы $C_{1}, C_{2}$ имеют точку пересечения $x_{i}$. Если после построения цикла $C_{2}$ в графе ещё остались непомеченные рёбра, аналогично описанному выше строится цикл $C_{3}$ и т.д., пока непомеченных рёбер не останется. Так как каждый из построенных циклов $C_{1}, C_{2}, \dots, C_{n}$ является эйлеровым подграфом графа $G$, то в силу связности графа $G$ получаем: граф $G$ - эйлеров.$\large_{\blacksquare}$
\par Для эйлеровых графов существует алгоритм Флёри для построения одного из возможных эйлеровых циклов:
\begin{enumerate}
    \item Выбираем вершину $x_{i}$, рассматриваем произвольное ребро и помечаем его как первое, переходим в вершину $x_{i + 1}$
    \item Циклически повторяем пункт 1 для $x_{i + 1}$, в качестве номеров используем последовательные натуральные числа
    \item Находясь в вершине $x_l$ не следует:
        \begin{enumerate}
            \item выбирать инцидентное ребро, второй концевой вершиной которого является предыдущая вершина, если есть другие варианты
            \item выбирать ребро-"перешеек", то есть такое ребро, удаление которого нарушит связность графа
        \end{enumerate}
\end{enumerate}
\par\textit{Пример:}
\begin{center}
    \includegraphics[width=0.4\paperwidth]{cycle}
\end{center}

\par\textbf{Теорема.} Связный неограф является эйлеровым тогда и только тогда, когда он может быть представлен объединением циклов, не пересекающихся по рёбрам.

%%%%%%%%%%%%%%%%%%%%%%%%%%%%%%% 4 file

\subsection*{Гамильтоновость в неографе}
\par\textbf{Опр.} Гамильтоновым обходом(циклом) называется в неографе называется обход(цикл), содержащий все вершины и проходящий через каждую из них только один раз.
\begin{itemize}
    \item Определить гамильтоновость графа сложнее, чем эйлеровость.
    \item Тут пример с пятиугольниками?
\end{itemize}
\par\textbf{Задача коммивояжёра.} Есть несколько пунктов, соединённых дорогами разной длины, и требуется обойти все, затратив меньшее количество сил.
\par Вычислительная сложность задачи нахождения Гамильтонова обхода в графе в общем случае: $O(n!)$
\par Задача коммивояжёра является **NP-полной**, то есть относится к классу задач, алгоритм решения которых можно применить для похожих задач.
\par Нет универсального алгоритма построения Гамильтонова цикла. Есть алгоритмы, упрощающие эту задачу при определённых требованиях к начальному графу:
\begin{enumerate}
    \item Алгоритм поиска Гамильтонова обхода в условиях теорем Дирака и Оре
    \item Алгоритм, улучшающий полный перебор за счёт использования динамического программирования
    \item Алгоритм, улучшающий полный перебор за счёт применения метода ветвей и границ
    \item Из ML: алгоритм поиска ближайшего соседа
\end{enumerate}

\par Критериев Гамильтоновости столь же простых как критерий эйлеровости не существует, но существует ряд теорем о достаточных условиях Гамильтоновости.
\par\textbf{Теорема Оре.} Дан неограф $G(X, U)$ порядка $n \geq 2$. Если для любой пары вершин $x_{i}, x_{j}$ выполняется неравенство $p(x_{i}) + p(x_{j}) \geq n$, то $G$ - Гамильтонов граф.

\subsection*{Эйлеровость в орграфе}
\par Дан орграф $G(X, U)$
\begin{enumerate}
    \item Если $G$ - эйлеров, то $\forall x \in X \; p^+(x_{i}) = p^-(x_{i})$
    \item Если $G$ - эйлеров, то он является объединением контуров, не пересекающихся по рёбрам.
\end{enumerate}

\par\textit{Пример:}
\begin{center}
    \includegraphics[width=0.35\paperwidth]{euler}
\end{center}

\par\textbf{Теорема.} Связный орграф $G(X, U)$ содержит открытый эйлеров путь тогда и только тогда, когда в нём найдётся 2 различных вершины $x, y \; (x \neq y)$ такие, что $p^-(x) = p^+(x) + 1$ и $p^-(y) = p^+(y) - 1$, а для всякой иной вершины $x_{i} \in X \setminus {x, y}$ верно $p^-(x_{i}) = p^+(x_{i})$
\subsection*{Гамильтоновость в орграфе}
\par Можно доказать гамильтоновость только в частном случае
\par\textbf{Теорема.} (одно из \textit{достаточных} условий)
\par Дан сильно связный орграф $G(X, U)$ без петель и кратных рёбер порядка $n \geq 2$. Если для любой пары различных несмежных вершин $x_{i}, x_{j}$ выполняется неравенство $p(x_{i}) + p(x_{j}) \geq 2n - 1$, то орграф $G$ содержит Гамильтонов контур.
\par Если теорема выполняется, то гарантированно существует гамильтонов обход, а если не выполняется, то он может как быть, так и не быть.

%%%%%%%%%%%%%%%%%%%%%%%%%%%%%%% 5 file

\section*{Деревья}
\par\textbf{Опр.} Неориентированное дерево - связный неограф без циклов.
\par\textbf{Опр.} Произвольный неограф без циклов называется лесом.
\subsection*{Свойства деревьев:}
\par $G(X, U)$ - неориентированные дерево
\par $|X| = n, |U| = m$
\begin{enumerate}
    \item $m = n - 1$
    \item Если $x_{i}, x_{j} \in X$, то их соединяет единственная простая цепь
        \begin{itemize}
            \item существование цепи следует из связности
            \item единственность из отсутствия циклов
        \end{itemize}
    \item Если $x_{i}, x_{j} \in X$ не смежны, то введение в дерево ребра $(x_{i} x_{j})$ даёт граф, содержащий ровно один цикл
    \item Всякое неориентированное дерево содержит по крайней мере 2 концевые вершины
    \item \textbf{Теорема Кайли.} Число различных деревьев, которые можно построить на $n$ вершинах равно $2^{n-2}$
\end{enumerate}

\par\textbf{Опр.} Орграф $G(X, U)$ называют ориентированным деревом(ордеревом, корневым деревом), если выполняются следующие условия:
\begin{itemize}
    \item существует ровно одна вершина (корень), не имеющая предшествующих ей: $p^+(x_{1}) = 0$
    \item для всех остальных вершин $p^+(x_{i}) = 1, \forall i \neq 1$
\end{itemize}

\par\textbf{Опр.} Висячие вершины дерева называются листьями.
\par\textbf{Опр.} Путь из корня в лист называется ветвью.
\par\textbf{Опр.} Длина наибольшей ветви называется высотой дерева.
\par\textbf{Опр.} Расстояние (число рёбер) от корня до вершины называется уровнем этой вершины.
\par\textbf{Опр.} Все вершины одного уровня называются ярусом.
\par Если из ордерева удалить корень, то оно распадётся на $k$ деревьев $\{ T_{1}, T_{2}, \dots, T_{k} \}$. На этом множестве деревьев можно задать отношение порядка. Если рекурсивно из этих поддеревьев снова удалить корни задать порядок и продолжить этот процесс, пока все поддеревья не станут вершинами, то получим упорядоченное множество всех вершин дерева. Это позволяет использовать деревья для описания иерархий объектов.
\par\textit{Примеры:}
\begin{itemize}
\item Разбор математических выражений
\item Файловая система
\item Описание сложных программных систем
\end{itemize}

\par\textbf{Опр.} Если полустепень исхода каждой отличной от листа вершины равна 2 и все листья дерева располагаются в одном ярусе, то дерево называется полным бинарным деревом.
\par Используя индукцию по высоте, можно доказать, что число листьев полного бинарного дерева высоты $h$ равно $2^h$:
\begin{enumerate}
    \item $h = 0, n = 2^0 = 1$
    \item $h = 1, n = 2^1 = 2$
    \item $h = k, n = 2^k$
    \item Чтобы получить дерево с высотой $h = k + 1$, каждому листу дерева высоты $h = k$ добавить 2 листа, тогда на $k + 1$ уровне получится $2 \cdot 2^k = 2^{k + 1}$ листьев
\end{enumerate}

\par\textbf{Теорема.} Произвольное бинарное дерево с $n$ листьями имеет высоту не меньше $\log_{2}n$

\par $\{ a_{1}, a_{3}, \dots, a_{n} \}$ - последовательность из $n$ элементов. Нужно ввести на ней отношение порядка - это задача сортировки.
\par В общем случае придётся рассмотреть $n!$ перестановок. Все сравнения приведут к построению бинарного дерева (дерева решений).
\begin{center}
    \includegraphics[width=0.7\paperwidth]{tree}
\end{center}

\par Дерево решений имеет $n!$ вершин, тогда сортировка даёт пути от корня к каждому листу. Число операций пропорционально высоте дерева - $\log_{2}(n!)$
\par Приближение для $n! \approx \sqrt{ 2\pi n } \large\left( \frac{n}{e} \right)^n$
\par Тогда получим минимально возможную сложность для алгоритма классической сортировки $O(n \log n)$

\par В общем случае ордерево - это связный, но не сильно связный орграф. Компонентами связности ордерева являются поддеревья на множестве вершин, образующие путь из корня в некоторый лист. - ???
\subsection*{Поиск в глубину}
\par $i$ - номер яруса
\par $k$ - номер ребра
\begin{enumerate}
    \item[0.] $i = 0, k = 0$
    \item Выбираем вершину, смежную текущей из яруса $i = i + 1$ по непомеченному ребру, новую вершину объявляем текущей, пройденное ребро помечаем числом $k = k + 1$
    \item Повторяем пункт 1 для текущей вершины
    \item Если текущая вершина является листом, то возвращаемся в смежную ей вершину ярусом выше $i = i - 1$ и переходим к пункту 1
    \item Если для текущей вершины нет ни одного инцидентного ей непомеченного ребра, возвращаемся в смежную ей вершину ярусом выше $i = i - 1$
    \item Алгоритм останавливается, когда помечены все рёбра и $i = 0$
\end{enumerate}

\break
\par\textit{Пример:}
\begin{center}
    \includegraphics[width=0.5\paperwidth]{dfs}
\end{center}
\par Поиск в глубину - обход дерева по ярусам
\par Оба алгоритма имеют одинаковую вычислительную сложность при обходе всех вершин
\par Если требуется найти конкретную вершину/ребро, то:
\begin{itemize}
    \item Поиск в глубину применяется для "широких"\:деревьев
	\item Поиск в ширину применяется для "узких"\:деревьев
\end{itemize}

\par Аналогичные подходы можно применять также для неориентированных деревьев

%%%%%%%%%%%%%%%%%%%%%%%%%%%%%%% 6 file

\section*{Остовы}
\par\textbf{Опр.} Граф $G'(X', U')$ называется остовным подграфом графа $G(X, U)$ $(X' = X, U' \subseteq U)$. Если остовный подграф является деревом, его называют остовным деревом или просто остовом.
\par То есть остов - это дерево построенное на множестве вершин графа, множество рёбер которого является подмножеством множества вершин исходного графа.
\par\textbf{Теорема.} Число рёбер произвольного неографа $G$, которое нужно удалить для получения остова, не зависит от порядка их удаления и равно $m - n + k$, где $m = |U|, n = |X|$, $k$ - число компонент связности.
\begin{itemize}
    \item $\nu(G) = m - n + k$ - циклический ранг (циклическое число), число рёбер, которые нужно удалить, чтобы получить остов
    \item $\nu^*(G) = n - k$ - коциклический ранг (коранг), число рёбер в остове
\end{itemize}

\subsection*{Задача Штейнера}
\par На плоскости задано $n$ точек, нужно соединить эти точки отрезками прямых так, что сумма длин этих отрезков была минимальной. Разрешается добавлять точки и длина определяется весом ребра.
\subsubsection*{Постановка задачи в теории графов:}
\par Дан граф $G_{0}(X_{0}, \varnothing)$ (задаёт $n$ точек). Требуется найти неограф $G(X, U)$ такой, что:
\begin{itemize}
    \item каждому ребру соответствует некоторое число (вес) $\geq 0$
    \item $G(X, U)$ - дерево
    \item множество вершин $X_{0}$ является подмножество нового множества вершин $X$: $X_{0} \subseteq X$
    \item сумма весов рёбер дерева $G$ должна быть минимальной из возможных
\end{itemize}

\par $G$ - неориентированное дерево, потому что каждую пару вершин должна соединять только одна цепь, иначе сумма весов будет не минимальной, так как, если существуют 2 цепи, возникают альтернативные пути и неоднозначность решения.
\par Решения задачи Штейнера в общем случае не существует.
\par Решения при некоторых ограничениях: алгоритм Прима и алгоритм Краскала.
\subsection*{Алгоритм Прима}
\par Ограничение: в ходе работы алгоритма не меняется множество вершин. В результате работы получается остовное дерево макс./мин. веса.
\par Дан граф $G(X, U, \Omega)$ ($\Omega$ - множество весов)
\par $|U| = |\Omega|$, $\Omega = \{ \omega_{1}, \dots, \omega_{n} \}$
\par В основе алгоритма лежит расширение исходного поддерева до остовного. Алгоритм итерационный, на каждой итерации число вершин дерева увеличивается не менее, чем на 1.
\par Множество вершин $X$ разбивается на 2 подмножества:
\par $X = X_{1} \cup X_{2}, X_{1} \cap X_{2} = \varnothing$
\par Введём понятие расстояния между $X_{1}$ и $X_{2}$:
\par $d(X_{1}, X_{2}) = \min \{ \omega(x_{i}, x_{j}): x_{i} \in X_{1}, x_{j} \in X_{2} \}$
\subsubsection*{Ход работы алгоритма:}
\begin{enumerate}
    \item $X_{1} = \{ x_{1} \}, X_{2} = X \setminus X_{1}, U = \varnothing$
    \item Находим ребро $(x_{i}, x_{j}): x_{i} \in X_{1}, x_{j} \in X_{2}, \omega(x_{i}, x_{j}) = d(X_{1}, X_{2})$ и полагаем $X_{1} = X_{1} \cup \{ x_{j} \}$, $X_{2} = X_{2} \setminus \{ x_{j} \}$, $U' = U' \cup \{ (x_{i}), x_{j} \}$
    \item Если $X_{1} = X$, то конец алгоритма, иначе повторить шаг 2.
\end{enumerate}

\par Если требуется найти остовное дерево наибольшего веса, то изменяем формулу расстояния (с $\min$ на $\max$).
\subsection*{Паросочетания}
\par Например, для графа из последовательно соединённых 6 вершин: $x_{1} - x_{2} - x_{3} - x_{4} - x_{5} - x_{6}$ возможны паросочетания:
\begin{itemize}
    \item $\{ (x_{1}, x_{2}), (x_{3}, x_{4}), (x_{5}, x_{6}) \}$
    \item $\{ (x_{2}, x_{3}), (x_{4}, x_{5}) \}$
\end{itemize}

\par\textbf{Опр.} Максимальным паросочетание называется паросочетание, в которое больше нельзя добавить ребро.
\par\textbf{Опр.} Паросочетание наибольшей мощности называется паросочетание из наибольшего возможного количества рёбер.
\par\textbf{Опр.} Совершенным пваросочетанием называется паросочетание, охватывающее все вершины.

\par Например, распределение задач между работниками может производиться как поиск паросочетаний в двудольном графе.

\end{document}