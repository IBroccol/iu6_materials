\documentclass[10pt]{article}
% for russian lang
\usepackage[T2A]{fontenc}
\usepackage[english, russian]{babel}
% end for russian lang

% config page
%\usepackage[paperheight=6in,
%   paperwidth=5in,
%   top=10mm,
%   bottom=20mm,
%   left=10mm,
%   right=10mm]{geometry}

\usepackage[a4paper, margin=20mm]{geometry}
\usepackage{amsfonts}
\usepackage{amsmath}
\usepackage{amssymb}
\usepackage{indentfirst}
\usepackage{csquotes}

\title{Теория множеств}
\author{ИУ6-25Б}
\date{2024}

\begin{document}
\maketitle

\section*{Наивная(канторовская) теория множеств}
\par\hfill\textit{"Множество - многое, мыслимое нами как единое целое."}
\par\hfill --- Георг Кантор
\par\begin{enumerate}
    \item Множества - $A, B, \dots, F_{1}, G_{10}$
    \item $a \in A$ - элемент $a$ принадлежит множеству $A$
    \item $A = B$ - ($\forall x \in A: x \in B$)
    \item Порядок элементов в множестве несущественен
    \item Элементы не могут повторяться
\end{enumerate}
\subsection*{Способы задания множества:}
\begin{enumerate}
    \item Множество задаётся набором элементов:
    \item[] $A = \{ 1, 2, a, c \}, B = \{ a, b, c, d \}, C = \{ a_{1}, \dots, a_{n} \}$
    \item Множество формируется из элементов другого множества:
    \item[] $A = \{ x: x \in \mathbb{R} \text{ and } \sqrt{x^2 + 1} < 3 \}$, где $P(x)$ - предикат (условие)
    \item[] $B = \left\{  x: x = \frac{\pi}{2} + 2\pi n, n \in \mathbb{N}  \right\}$
    \item Множество формируется их элементов этого же множества:
    \item[] $F = \{ f(i): f(0) = 1, f(1) = 1, f(i) = f(i - 1) + f(i - 2), i = 2, 3\dots \}$
\end{enumerate}

\par\textbf{Опр.} Множество, состоящее из конечного числа элементов, называется конечным, из бесконечно числа элементов - бесконечным.
\par\textbf{Опр.} Множество, не содержащее элементов, называется пустым множеством ($\varnothing$).
\par\textbf{Опр.} Множество, состоящее из элементов, образующих все возможные множества данной задачи или класса задач, называется универсальным: $U$
\par\textbf{Опр.} Множество $B$ называется подмножеством множества $A$, если каждый элемент $B$ является элементом $A$.
\par\textbf{Обоз.} $B \subseteq  A$ - нестрогое включение ("$A$ включает $B$" или "$B$ содержится в $A$")
\par По опр.:
\begin{itemize}
\item $\varnothing \subseteq A$
\item $A \subseteq U$
\item $(A = B) \iff (A \subseteq B \text{ и } B \subseteq A)$
\item $B \subseteq A, A \neq B \Rightarrow B \subset A$ - строгое включение, $B$ - собственное подмножество $A$
\end{itemize}

\par\textbf{Опр.} Булеан - множество всех подмножеств $A$
\par\textbf{Обоз.} $2^A, 2^{2^A}\dots$
\par\textbf{Свойства включения множеств:}
\begin{enumerate}
    \item $A \subseteq A$
    \item $A \subseteq B, B \subseteq C \Rightarrow A \subseteq C$
\end{enumerate}

\par $|A|$ - мощность множества $A$ по Кантору
\par $|A|=|B|=n=const$
\par $|A|=|B| \not\Rightarrow A=B$
\par $|A| < \infty$ - конечное множество
\par $|A| = \infty$ - бесконечное множество
\par Если $A$ - бесконечное множество, то оно равномощно некоторому подмножеству.
\par Множество, у которого отсутствует равномощное ему собственное подмножество, называется конечным.
\par\textbf{Теорема.} Множество, имеющее бесконечное подмножество, само бесконечно.
\par\textbf{Следствие.} Все подмножества конечного множества конечны.

\par $|A \cup B| = |A| + |B| - |A \cap B|$
\par $|A \cup B \cup C| = |A| + |B| + |C| - |A \cap B| - |B \cap C| - |A \cap C| + |A \cup B \cup C|$
\par $|A_{1} \cup A_{2} \cup \dots \cup A_{n}| = |A_{1}| + \dots + |A_{n}| - (|A_{1} \cap A_{2}| + |A_{1} \cap A_{2}| + \dots + |A_{n - 1} \cap A_{n}|) + (|A_{1} \cap A_{2} \cap A_{3}| + \dots + |A_{n - 2} \cap A_{n - 1} \cap A_{n}|) - \dots + (-1)^{n-1} |A_{1} \cap A_{2} \cap \dots \cap A_{n}|$

\section*{Прямое(декартово) произведение множеств}
\par $a \in A, b \in B$
\par $\{ a, b \} = \{ b, a \}$ - неупорядоченная пара
\par $(a, b) \neq (a, b)$ - упорядоченная пара
\par 
\par $A_{1}, A_{2}, \dots, A_{n}$
\par $a_{1} \in A_{1}, a_{2} \in A_{2}, \dots, a_{n} \in A_{n}$
\par $(a_{1}, a_{2}, \dots, a_{n})$ - кортеж
\par\textbf{Опр.} Множество всех кортежей длины $n$ на множествах $A_{1}, .., A_{n}$ называется прямым (декартовым) произведением этих множеств.
\par\textbf{Обоз.} $A_{1} \times A_{2} \times \dots \times A_{n} = \{ (x_{1}, .., x_{n}): x_{1} \in A_{1}, \dots, x_{n} \in A_{n} \}$

\par Если $A_{1} = A_{2} = \dots = A_{n}$ , то $A \times A \times \dots \times A = A^n$ - $n$-степень множества $A$
\begin{itemize}
    \item $n = 2: A^2$ - декартов квадрат
    \item $n = 1: A^1 = A$
\end{itemize}

\par $A \times B = \{ (x, y): x \in A, y \in B \}$
\par $A \times B \neq B \times A$
\par\textit{Пример:}
\begin{itemize}
    \item $A = \{  a_{1}, a_{2} \}$
    \item $B = \{ b_{1}, b_{2}, b_{3} \}$
    \item $A \times B = \{ (a_{1}, b_{1}), (a_{1}, b_{2}), (a_{1}, b_{3}), (a_{2}, b_{1}), (a_{2}, b_{2}), (a_{2}, b_{3})\}$
    \item $B \times A = \{ (b_{1}, a_{1}), (b_{1}, a_{2}), (b_{2}, a_{1}), (b_{2}, a_{2}), (b_{3}, a_{1}), (b_{3}, a_{2})\}$
\end{itemize}

\par $|A \times B| = |A||B|$
\par $|A_{1} \times A_{2} \times \dots \times A_{n}| = |A_{1}||A_{2}|\dots|A_{n}|$
\subsection*{Геометрический смысл}
\par $A = [a_{1}, a_{2}], B = [b_{1}, b_{2}]$ - отрезки
\par Геометрический смысл прямого (декартова) произведения заключается в том, что $A \times B$ - множество координат всех точек заштрихованного прямоугольника таких, что абсциссы $\in A$ и ординаты $\in B$
\par $|A^n| = |A|^n$
\par\textbf{Свойства декартова произведения:}
\begin{enumerate}
    \item $A \times (B \cup C) = (A \times B) \cup (A \times C)$
    \item $A \times (B \cap C) = (A \times B) \cap (A \times C)$
    \item $A \times \varnothing = \varnothing \times A = \varnothing$
\end{enumerate}
\par Доказываются методом включений

\section*{Отображения бинарноых отношений}
\par Обычно используются буквы $f, g, h\dots$ как для функций.

\par\textbf{Опр.} Отображение $f: A \rightarrow B$ из множества $A$ в множество $B$ задано, если каждому элементу $x \in A$ соответствует элемент $y \in B$.
\par\textbf{Обоз.} $f: A \rightarrow B$

\par\textbf{Опр.} $y$ называется образом элемента $x$ при отображении $f$ ($y = f(x)$).
\par Каждое отображение задаёт множество упорядоченных пар таких, что $$\{ (x, y): x \in A, y = f(x) \} \subseteq A \times B$$
\par Когда для отображения $f$ могут существовать несколько различных элементов из $A$, имеющих один и тот же образ $y_{0}$ , такие элементы $x$ называют **прообразами** элемента $y_{0}$ при отображении $f$.
\par $\{ x_{1}, x_{2}, x_{2} \} \subset A$
\par $f(x_{1}) = f(x_{2}) = f(x_{3}) = y_{0}$
\par\textit{Пример:}
\begin{itemize}
    \item $y = \cos x \qquad 0 \leq y_{0} \leq 1$
    \item $\{ x: x = \arccos{y} \pm 2\pi n, n \in \mathbb{N} \}$
\end{itemize}

\par\textbf{Опр.} Областью значения отображения $f$ называется множество всех конечных элементов $y \in B$, для которых найдётся $x \in A: y = f(x)$.
\begin{enumerate}
    \item Отображение $f: A \rightarrow B$ называется инъективным (инъекция), если для каждого $y$ из области значения отображения $f$ существует единственный прообраз.
    \item[] $y_{1} = f(x_{1}), y_{2} = f(x_{2})$
    \item[] $(y_{1} = y_{2}) \Rightarrow (x_{1} = x_{2})$
    \item Отображение $f: A \rightarrow B$ называется сюръективным (сюръекция), если область значений отображения $f$ полностью совпадает с множеством $B$.
    \item Отображение $f: A \rightarrow B$ называется биективным (биекция), если оно одновременно инъективно и сюръективно.
    \item[] \textit{Пример:} $y = \text{arctg } x$ - биекция на $\left(-\frac{\pi}{2}; \frac{\pi}{2}\right)$
\end{enumerate}
\par\textit{Примечание:} Смещение каждой точки окружности при повороте на угол вокруг центра есть биекция точек окружности на саму себя - автоморфизм.

\subsubsection*{Обобщение понятия отображения:}
\begin{enumerate}
    \item Если образ $y \in B$ определён не для каждого $x \in A$, имеет место частичное отображение.
    \item Если отображение неоднозначное (некоторым элементам $x \in A$ соответствует не по одному элементу $y \in B$, то есть несколько образов), то имеет место соответствие множества $A$ множеству $B$.
\end{enumerate}

\par $\rho \subseteq A \times B$ - задание соответствия из $A$ в $B$
\par $\rho = \varnothing$ - частный случай
\par $\rho = A \times B$ - универсальное соответствие

\par $a, b: a \in A, b \in B$
\par $(a, b) \in \rho$ - упорядоченная пара входит в соответствие $\rho$
\par $Def(\rho)$ - область определения соответствия $\rho$, множество всех первых компонентов упорядоченных пар, составляющих соответствие $\rho$
\par $Def(\rho) = \{ x: (\exists y \in B), (x, y) \in \rho \}$
\par $Res(\rho)$ - область значений соответствия $\rho$, множество всех вторых компонентов упорядоченных пар, составляющих соответствие $\rho$
\par $Res(\rho) = \{ y: (\exists x \in A), (x, y) \in \rho \}$
\par
\par\textbf{Опр.} Сечением соответствия $\rho$ по элементу $x_0 \in A$ называется множество $\rho(x_0)$ из вторых компонентов пар соответствия $\rho$ таких, что первым компонентом является $x_0$
\par\textbf{Обоз.} $\rho(x_0) = \{ y: (x_0, y) \in \rho \}$

\par\textbf{Опр.} Сечением соответствия $\rho$ по множеству $E \subseteq A$ называется множество всех вторых компонентов пар соответствия $\rho$ таких, что первый компонент входит в множество $E$
\par\textbf{Обоз.} $\rho(E) = \{ y: (x, y) \in \rho, x \in E \}$

\par\textbf{Опр.} Обратным соответствием $\rho^{-1} \subseteq B \times A$ называется соответствие, определяемое как множество пар $(y, x)$ таких, что $(x, y) \in \rho$
\par\textbf{Обоз.} $\rho^{-1} = \{ (y, x): (x, y) \in \rho \}$
\par $(\rho^{-1})^{-1} = \rho$

\begin{itemize}
    \item Если задано отображение $f: A \rightarrow B$, то оно является соответствием
    \item Обратное ему отображение $f^{-1}: B \rightarrow A$ в общем случае соответствием не является
\end{itemize}

\section*{Бинарные отношения}
\par\textbf{Опр.} $R \subseteq A \times A$ или $R \subseteq A^2$ - бинарное отношение на множестве $A$
\par\textit{Пример:}
\begin{itemize}
    \item $x, y \in \mathbb{N}$
    \item $x \leq y$ - инфиксная запись отображения
    \item $(x, y) \in$ "$\leq$" - имя бинарного отношения
    \item В общем виде: $xRy$ или $(x, y) \in R$
\end{itemize}

\par\textbf{Опр.} Если $R$ - бинарное отношение, то обратное ему соответствие есть бинарное отношение $R^{-1}$ на том же множестве $A$
\par\textbf{Опр.} Бинарное отношение $R$, в каждой паре которого компоненты совпадают, равномощное множеству $A$ называется диагональю множества $A$
\par\textbf{Обоз.} $id_A$
\par Диагональ является отображением
\subsection*{Способы задания бинарных отношений:}
\begin{enumerate}
    \item Перечисление пар:
        \begin{itemize}
            \item $A = \{ a_1, a_2, a_3 \}$
            \item $R = \{ (a_1, a_1), (a_1, a_2), (a_2, a_3), (_1, a_3) \}$
        \end{itemize}
    \item Таблица: (число столбцов равно $Def(R)$)
        \par \begin{tabular}{|c|c|c|}
            \hline
            $R(Def(R))$ & $a_1$ & $a_2$ \\
            \hline
            $R(Res(R))$ & $\{ a_1, a_2, a_3\}$ & $\{ a_3 \}$ \\
            \hline
        \end{tabular}
    \item Матрица бинарного отношения: (квадратная порядка $n = |A|$, $r_{ij} = 1 \; (a_i, a_j) \in R$):
        \par \begin{tabular}{|c|c|c|c|}
            \hline
             & $a_1$ & $a_2$ & $a_3$ \\
            \hline
            $a_1$ & $1$ & $1$ & $1$ \\
            \hline
            $a_2$ & & & $1$ \\
            \hline
            $a_3$ & & & \\
            \hline
        \end{tabular}
    \item Задание двудольным графом
\end{enumerate}

\subsection*{Способы задания соответствия:}
\par $\rho \subseteq A \times B$
\begin{enumerate}
    \item Перечисление пар:
        \begin{itemize}
            \item $A = \{ a_1, a_2, a_3 \}$
            \item $B = \{ b_1, b_2 \}$
            \item $\rho = \{ (a_1, b_1), (a_1, b_2), (a_2, b_2), (a_3, b_2) \}$
        \end{itemize}
    \item Таблица:
        \par \begin{tabular}{|c|c|c|c|}
            \hline
            $Def(\rho)$ & $a_1$ & $a_2$ & $a_3$ \\
            \hline
            $\rho(Res(\rho))$ & $\{ b_1, b_2 \}$ & $\{ b_2 \}$ & $\{ b_2 \}$ \\
            \hline
        \end{tabular}
    \item Матрица (сетка) $m \times n$, где $n = |A|, m = |B|, \rho_{ij} = 1 \; (a_i, a_j) \in \rho$:
        \par \begin{tabular}{|c|c|c|}
            \hline
             & $b_1$ & $b_2$ \\
            \hline
            $a_1$ & $1$ & $1$ \\
            \hline
            $a_2$ & & $1$ \\
            \hline
            $a_3$ & & $1$ \\
            \hline
        \end{tabular}
    \item Двудольный граф
\end{enumerate}
\subsection*{Свойства бинарных отношений}
\par Пусть дано множество $A, \; |A| = n, \; R \subseteq A^2$
\begin{enumerate}
    \item Рефлексивность:
        \begin{itemize}
            \item \textbf{Опр.} Отношение $R$ называется \textbf{рефлексивным}, если $\forall x \in A: xRx$, то есть $(x, x) \in R$ или $id_{A} \in R$
            \item[] Все элементы на главной диагонали матрицы такого отношения равны 1.
            \item \textbf{Опр.} Если $id_{A}$ полностью отсутствует в $R$, то такое отношение называется \textbf{иррефлексивным} (\textbf{антирефлексивным})
            \item \textbf{Опр.} Если часть элементов элементов $id_A$ присутствует в $R$, а часть нет, то такое отношение называется \textbf{нерефлексивным}
            \item \textit{Пример:}
            \item[] "=" - рефлексивное отношение
            \item[] "$\neq$"- иррефлексивное отношение
        \end{itemize}
    \item Симметричность:
        \begin{itemize}
            \item \textbf{Опр.} Отношение $R$ называется \textbf{симметричным}, если $(x, y) \in R: (y, x) \in R$  ($xRy \Rightarrow yRx$, $R = R^{-1}$)
            \item Матрица такого отношения симметрична относительно главной диагонали.
            \item\textbf{Опр.} Если хотя бы для одной пары условие симметричности ен выполняется, то такое отношение называется \textbf{несимметричным}
        \end{itemize}
    \item Антисимметричность:
        \begin{itemize}
            \item Более жёсткое требование, чем несимметричность
            \item\textbf{Опр.} Отношение $R$ называется \textbf{антисимметричным}, если $(xRy \text{ и } yRx) \Rightarrow x = y$
            \item Не конфликтует с рефлексивностью
        \end{itemize}
    \item Транзитивность:
        \begin{itemize}
            \item \textbf{Опр.} Отношение $R$ называется \textbf{транзитивным}, если $\forall x, y, z \in A \; (xRy \text{ и } yRz) \Rightarrow  xRz$
            \item \textbf{Опр.} Если хотя бы для одного набора $x, y, z \in A \; (xRy \text{ и } yRz) \not\Rightarrow xRz$, такое отношение называется \textbf{нетранзитивным}
        \end{itemize}
    \item Плотность:
        \begin{itemize}
            \item \textbf{Опр.} Отношение $R$ называется плотным, если $\forall x, y \in A: xRy \; \exists z \in A: xRz \text{ и } zRy$
            \item Для любых различных элементов множества $R$ можно указать третий элемент из $R$, который "встраивается" между первыми двумя
        \end{itemize}
\end{enumerate}

\section*{Классы бинарных отношений}
\begin{tabular}{|c|c|c|c|c|c|}
    \hline
    Отношение\textbackslash Свойства & Иррефл-ть & Рефл-ть & Сим-ть & Антиссим-ть & Транз-ть \\
    \hline
    Эквивалентность & & + & + & & + \\
    \hline
    Толерантность & & + & + & & \\
    \hline
    Порядок (частичный порядок) & & + & & + & + \\
    \hline
    Пред(варительный) & & + & & & + \\
    порядок(квазипорядок) & & & & & \\
    \hline
    Строгий порядок & + & & & + & + \\
    \hline
    Строгий предпорядок & + & & & & + \\
    \hline
\end{tabular}

\section*{Отношение эквивалентности}
\par Пусть $A$ - некоторое множество
\par\textbf{Опр.} Семейство попарно не пересекающихся множеств $C: i = \overline{1, n}$ называется разбиением множества $A$, если их объединение даёт $A$
\par $$\bigcup_{i=1}^n C_{i} = A$$
\par\textit{Пример:}
\begin{enumerate}
    \item $A = \{ 1, 2, 7, 8, 12, 15 \}$, $C_1 = \{ 1, 2, 12, 15 \}$, $C_{2} = \{ 7, 8 \}$
    \item[] $A = C_{1} \cup C_{2}$
    \item[] $C_{1} \cap C_{2} = \varnothing$
    \item $A = [1, 5]$
    \item $C_{1} = [1, 5), C_{2}=[2, 3.5), C_{3} = [3.5, 5]$
\end{enumerate}

\par Пусть $R$ - отношение эквивалентности на множестве $A$ и $x \in A$
\par\textbf{Опр.} Классом эквивалентности $[x]_{R}$ по отношению $R$  называется множество всех вторых компонентов пар отношения $R$, у которых первым компонентом является $x$
$[x]_{R} = \{ y \in A: xRy \}$ - сечение эквивалентности по $x$
\par Так как $R$ - эквивалентность и она рефлексивна, то класс эквивалентности всегда не пустой: $[x]_{R} \not = \varnothing$
\par $\forall x \in A \; xRx$, то есть $id_{A} \in R$
\par\textit{Пример:} множество всех прямых на плоскости, параллельных данной

\par\textbf{Теорема.} Для любого отношения эквивалентности на множестве $A$ множество классов эквивалентности образует разбиение множества $A$
\par\textbf{Следствие:} Любое разбиение множества задаёт на нём отношение эквивалентности, для которого классы разбиения образуют....
\par То есть любая эквивалентность определяет единственное разбиение множества и наоборот

\par\textbf{Опр.} Множество всех классов эквивалентности по данному отношению $R$ называется фактор-множеством множества $A$ по отношению $R$
\par\textbf{Обоз.} $A / R$
\par\textit{Пример:}
\begin{itemize}
    \item $A = \{ a, b, c, d, e\}$
    \item $[a]_{R} = \{ a, b \} = c_{1}$
    \item $[b]_{R} = \{ a, b \}$
    \item $[c]_{R} = \{ c \} = c_{2}$
    \item $[d]_{R} = \{ d, e \} = c_{3}$
    \item $[e]_{R} = \{ d, e \}$
    \item $A = c_{1} \cup c_{2} \cup c_{3}$
    \item $A / R = \{ c_{1}, c_{2}, c_{3} \}$
\end{itemize}
\par Существует связь между понятиями эквивалентности, разбиения и отображения. $\forall R \subseteq A^2$ можно задать отображение множества $A$ в его фактор-множество $A / R$
\par Если считать $x \in A, f(x) = [x]_{R}$ , то получим, что каждому элементу $x \in A$ отображение $f$ сопоставляет единственный класс эквивалентности, содержащий этот элемент. Заметим, что отображение $f$ - сюръективное.
\par Любое отображение однозначно определяет некоторое отношение эквивалентности.
\par\textbf{Теорема.} Пусть $f$ - произвольное отображение, отношение $R$ на множестве $A$, для которого $(x, y) \in R$ возможно тогда и только тогда, когда $f(x) = f(y)$, является отношение эквивалентности. Причём существует биекция фактор-множества $A / R$ на множество $f(A)$ ($A / R \leftrightarrow f(A)$)
\par Из теоремы \textbf{не следует}, что между $f$ и $R$ существует взаимно однозначное соответствие, два разных отображения могут задавать одно и то же разбиение множества $A$
\begin{itemize}
    \item $f_{1}: A \to B_{1}$
    \item $f_{2}: A \to B_{2}$
\end{itemize}

\section*{Упорядоченные множества. Отношения порядка.}
\par\textbf{Опр.} Множество с заданным на нём отношением порядка называется упорядоченным
\par\textbf{Обоз.} $(A, R)$
\par\textit{Пример:} $(A, \leq)$
\par Каждому отношению порядка на множестве $A$ можно сопоставить следующие отношения:
\begin{enumerate}
    \item Отношение строго порядка: $<$
        \begin{itemize}
            \item удалить $id_A$ из классического $\le$
            \item $\forall x, y \in A \; x < y \Leftrightarrow x \le y, x \not = y$
        \end{itemize}
    \item Отношение, двойственное классическому порядку: $\ge$
    \item[] $\forall x, y \in A \; x \le y \Leftrightarrow y \le x$
    \item Отношение, двойственное к строгому порядку: $<$
    \item[] $\forall x, y \in A \; x > y \Leftrightarrow y \le x, x \not = y$
    \item Доминирование: $x \mathrel{<\mkern-8mu\mid\mkern8mu} y$ ($y$ доминирует над $x$)
        \begin{itemize}
            \item $x \mathrel{<\mkern-8mu\mid\mkern8mu} y$ если $x < y$ и $\not \exists z \in A: x < z < y$
            \item Не существует элемента, который можно встроить между $x$ и $y$  по отношению строго меньше
            \item Доминирование иррефлексивно, антисимметрично и нетранзитивно
        \end{itemize}
\end{enumerate}

\par\textbf{Опр.} 2 элемента $x, y$ называются сравнимыми по отношению порядка "не больше", если $x \leq y$ или $y \leq x$, иначе - несравнимыми элементами по отношению порядка "не больше"
\par\textbf{Опр.} Упорядоченное множество $(A, \leq)$, все элементы которого попарно сравнимы, называется линейно упорядоченным, а соответствующее отношение называется линейным порядком
\begin{itemize}
    \item Линейный порядок на множестве $A$ может быть перенесён на любое непустое подмножество $A$
    \item Если порядок на $A$ - линейный, то порядок на $B \subset A$ - тоже линейный
\end{itemize}

\par\textbf{Опр.} Любое подмножество попарно несравнимых элементов множества $A$ называется антицепью
\par\textbf{Опр.} Элемент $a \in (A, \leq)$ называется наибольшим элементом множества $A$, если $\forall x \in A \; x \leq a$
\par\textbf{Опр.} Элемент $b \in (A, \leq)$ называется максимальным элементом множества $A$, если $\forall x \in A \; x \leq b$ или $x$ и $a$ несравнимы
\par\textbf{Теорема.} Наибольший (наименьший) элемент упорядоченного множества, если он существует, является единственным
\par\textbf{Доказательство.}
\par Пусть $(A, \leq)$. Предположим, что в нём 2 максимальных элемента $a_{1}, a_{2}$. Тогда $\forall x \in A \; x \leq a_{1}, x \leq a_{2}$
Так как $a_{1}, a_{2} \in A$, то $a_{1} \leq a_{2}$ и $a_{2} \leq a_{1} \Rightarrow a_{1} = a_{2} \Rightarrow$ наибольший элемент единственный $_{\blacksquare}$
\par\textbf{Опр.} Пусть $(A, \leq), B \subset A$. Элемент $a \in A$ называется верхней (нижней) гранью множества $B$, если $\forall x \in B \; x \leq a \; (x \geq a)$
\par\textbf{Опр.} Наименьший элемент множества всех верхних граней множества $B$ называется точной верхней гранью множества $B$
\par\textbf{Обоз.} $\sup B$ - supremum
\par\textbf{Опр.} Наибольший элемент множества всех нижних граней множества $B$ называется точной нижней гранью множества $B$
\par\textbf{Обоз.} $\inf B$ - infinum

\par\textbf{Теорема.} Всякое ограниченное сверху непустое множество имеет верхнюю грань, а всякое ограниченное снизу непустое множество имеет нижнюю грань
\par\textbf{Опр.} $(A, \leq)$ называется вполне упорядоченным, если любое его непустое подмножество имеет наименьший элемент
\par Если есть $(A, \leq)$ и есть свойство, доказанное для этого порядка, то это свойство будет справедливо для двойственного порядка, если:
\begin{enumerate}
    \item заменить $\leq$ на $\geq$ и наоборот
    \item максимальный элемент заменить минимальным
    \item $\inf$ заменить на $\sup$ и наоборот
\end{enumerate}

\par Конечное упорядоченное множество малой мощности удобно показать с помощью диаграммы Хассе
\vspace{5mm}
\par $\{ x_{i} \}, i \in \mathbb{N}$ - последовательность элементов
\par\textbf{Опр.} Последовательность элементов $(A, \leq)$ $\{ x_{i} \}, i \in \mathbb{N}$ называется неубывающей, если $\forall i \in \mathbb{N} \; x_{i} \leq x_{i + 1}$
\par\textbf{Опр.}  Элемент $x \in (A, \leq)$ называется точной верхней гранью последовательности $\{ x_{i} \}$, если он является точной верхней гранью множества всех членов последовательности.
\par\textbf{Опр.} Упорядоченное множество $(A, \leq)$ называется индуктивным, если:
\begin{enumerate}
    \item оно содержит наименьший элемент
    \item всякая неубывающая последовательность $\{ x_{i} \}$ элементов этого множества имеет точную верхнюю грань
\end{enumerate}
\vspace{5mm}
\par\textbf{Опр.} Пусть имеется 2 индуктивных упорядоченных множества $(A_{1}, \leq)$ и $(A_{2}, \preceq)$. Отображение $f: A_{1} \to A_{2}$ называется непрерывным, если для любой неубывающей последовательности элементов множества $A_{1}$: $a_{1}, a_{2}, \dots, a_{n}, \dots$ образ её точной верхней грани равен точной верхней грани последовтельности $f(a_{1}), f(a_{2}), \dots, f(a_{n}), \dots$ $$f(\sup \{ a_{n} \}) = \sup \{ f(a_{n}) \}$$
\par\textbf{Опр.} Отображение $f: A_{1} \to A_{2}$ называется монотонным, если $\forall a, b \in A_{1} \quad a \leq b: f(a) \preceq f(b)$
\par\textbf{Теорема.} Всякое непрерывное отображение одного индуктивного упорядоченного множества в другое является монотонным
\par\textbf{Опр.} Элемент $a \in (A, \leq)$ называется неподвижной точкой отображения $f: A \to A$, если $f(a) = a$
\vspace{5mm}
\par\textbf{Теорема о неподвижной точке.}
\par Любое непрерывное отображение $f: A \to A$ индуктивного упорядоченного множества $A$ в себя имеет наименьшую неподвижную точку
\vspace{5mm}
\par Уравнение $f(x) = x$ имеет решение $x_{0} \in A \quad x_{0} = f(x_{0})$
\par Множество всех решений уравнения образует множество всех неподвижных точек и оно имеет наименьший элемент.
\par\textit{Пример:}
\begin{itemize}
    \item[] Множество $(A, \leq)$: $A = [0, 1]$ - индуктивно
    \item[] Отображение: $f: A \to A$
    \item[] $f(x) = \frac{1}{2}x + \frac{1}{4}$
    \item[] $x_{0} = f(x_{0})$, $x_{0} = 0$
    \item[] $f^0(0) \neq 0$
    \item[] $f^1(0) = \frac{1}{4}$
    \item[] $f^2\left( \frac{1}{4} \right) = \frac{3}{8}$
    \item[] $f^3\left( \frac{3}{8} \right) = \frac{7}{16}$
    \item[] $f^4\left( \frac{7}{16} \right) = \frac{15}{32}$
    \item[] $0 \leq \frac{1}{4} \leq \frac{3}{8} \leq \frac{7}{16} \leq \frac{15}{32}$
    \item[] Путём бесконечного числа итераций получается неубывающая последовательность$$\lim_{ n \to \infty } \frac{2^n - 1}{2^{n+1}} = \frac{1}{2} = x_{наим}$$
    \item[] $x_{наим} = f(x_{наим})$
    \item[] $\frac{1}{2} = f\left( \frac{1}{2} \right) = \frac{1}{2} \frac{1}{2} + \frac{1}{4} = \frac{1}{2}$ - верно
    \item[] Наименьшая неподвижная точка - $\frac{1}{2}$
\end{itemize}

\section*{Мощность множеств}
\par\textbf{Опр.} Множество $A$ называется равномощным множеству $B$, если существует биекция $f: A \leftrightarrow B$
\par\textbf{Обоз.} $A \sim B$
\begin{itemize}
    \item Из равномощности $A$ и $B$ следует, что $\exists f^{-1}: B \leftrightarrow A$
    \item Из определения равномощности и свойств биекции следует, что $A \sim A$
    \item Равномощность рефлексивна, симметрична и транзитивна, то есть относится к классу эквивалентности
    \item Равномощность - это не то же самое, что равенство множеств
    \item Если обозначить класс эквивалентности $|A|$ по отношению равномощности, то получим мощность множества $A$
\end{itemize}

\par\textbf{Опр.} Мощность множества $A$ - класс эквивалентности по отношению равномощности
\begin{itemize}
    \item Если $|A| = |B|$,  $A = \{ a_{1}, \dots, a_{n}\}$, и $B = \{ b_{1}, \dots, b_{m}\}$, то $m = n$
    \item Если множество конечно, оно не будет равномощно ни одному своему собственному подмножеству
\end{itemize}

\par\textbf{Теорема.} Если $A$ - некоторое множество и имеет место инъекция из $A$ в $A$, то она является сюръекцией и биекцией.
\par На примере счётных множеств:
\par\textbf{Опр.} Любое множество, равномощное множеству $\mathbb{N}$ называется счётным
\par\textbf{Опр.} Биекцию множества $M$ с множеством $\mathbb{N}$ называют нумерацией: $\varphi: M \leftrightarrow \mathbb{N}$

\par $|2^{\mathbb{N}}| = \mathfrak{C}$ - континуум
\par $|\mathbb{N}| = \aleph_{0}$ - алеф-нуль

\subsection*{Сравнение мощностей бесконечных множеств.}
\par\textbf{Опр.} Даны множества $A$ и $B$, считается, что $|A| \leq |B|$, если $A$ равномощно некоторому подмножеству $B$
\par $|A| \leq |B| \land |A| \geq |B| \implies A \sim B$
\par\textbf{Опр.} $|A| < |B|$, если $|A| \neq |B|$ и $\exists C \subset B: A \sim C$
\par Сравнение мощностей транзитивно: $|A| < |B| \land |B| < |C| \implies |A| < |C|$

\section*{Свойства счётных множеств (теоремы)}
\begin{enumerate}
    \item Любое бесконечное множество содержит счётное подмножество
    \item В любом бесконечном множестве можно выделить 2 не пересекающихся счётных подмножества
    \item Любое подмножество счётного множества конечно или счётно
    \item Объединение любого конечного или счётного семейства счётных множеств является счётным
    \item Объединение конечного и счётного множества счётно
    \item Следующие множества равномощны:
        \begin{enumerate}
            \item $[0, 1] \in \mathbb{R}$
            \item $(0, 1) \in \mathbb{R}$
            \item $[a, b] \in \mathbb{R}$
            \item $(a, b) \in \mathbb{R}$
            \item $\mathbb{R}$
            \item $2^{\mathbb{N}}$
        \end{enumerate}
    \item Теорема о квадрате. Для произвольного множества $A$ верно, что $|A| \sim |A^2|$
    \item Теорема Кантора-Бернштейна. Для любых множеств $A$ и $B$ верно одно из трёх:
        \begin{enumerate}
            \item[1)] $|A| < |B|$
            \item[2)] $|A| > |B|$
            \item[3)] $|A| = |B|$
        \end{enumerate}
    \item Для любого множества $A$ верно неравенство: $|2^\mathbb{N}| > |A|$
    \item[] Для каждого множества существует множество большей мощности - булеан.
    \item Следствие из теоремы о квадрате. Множество рациональных чисел $\mathbb{Q}$ счётно.
        \begin{itemize}
            \item Любое рациональное число можно представить в виде дроби $\frac{a}{b}$ или пары взаимно простых чисел $(a, b)$
            \item Тогда $\mathbb{Q} \sim$ некоторому подмножеству $\mathbb{Z}^2$
            \item Согласно теореме о квадрате $\mathbb{Z} \sim \mathbb{Z}^2$
            \item Так как $\mathbb{Z}$ и $\mathbb{Z}^2$ счётны, а любое подмножество счётного множества конечно или счётно, то $\mathbb{Q}$ - счётно.
        \end{itemize}
\end{enumerate}

\end{document}