\documentclass[11pt]{article}
% for russian lang
\usepackage[T2A]{fontenc}
\usepackage[english, russian]{babel}
% end for russian lang

% config page
%\usepackage[paperheight=6in,
%   paperwidth=5in,
%   top=10mm,
%   bottom=20mm,
%   left=10mm,
%   right=10mm]{geometry}

\usepackage[a4paper, total={6in, 8in}, margin=2cm]{geometry}
\usepackage{amsfonts}
\usepackage{amsmath}
\usepackage{amssymb}
\usepackage{indentfirst}
\usepackage{csquotes}

\title{Теория для РК2 по ЛАиФНП}
\author{ИУ6-25Б}
\date{2024}

\begin{document}

\maketitle

\section*{1. Дать определение окрестности и открытого множества в $\mathbb{R}^{n}$.}
\par\textbf{Опр.} $\varepsilon$-окрестностью точки $a \in \mathbb{R}^{n}$ называется множество $U_{\varepsilon}(a)$ всех точек $x \in \mathbb{R}^{n}$, расстояние от которых до точки $a$ меньше $\varepsilon$.
\par То есть $U_{\varepsilon}(a) = \{ x \in \mathbb{R}^{n} \; | \; \rho(x, a) < \varepsilon \}$
\par Для проколотой: $\mathring{U}_{\varepsilon}(a) = \{ x \in \mathbb{R}^{n} \; | \; 0 < \rho(x, a) < \varepsilon \}$ 
\par\textbf{Опр.} Множество $A \subset \mathbb{R}^{n}$ называется открытым, если все его точки внутренние.
\section*{2. Дать определение предельной точки, граничной точки множества и замкнутого множества в $\mathbb{R}^{n}$.}
\par\textbf{Опр.} Точка $a \in \mathbb{R}^{n}$ называется граничной точкой множества $A \subset \mathbb{R}^{n}$, если любая окрестность $U_{\varepsilon}(a)$ содержит и точки из $A$, и точки из $\mathbb{R}^{n} \setminus A$.
\par\textbf{Опр.} Точка $a \in \mathbb{R}^{n}$ называется предельной точкой множества $A \subset \mathbb{R}^{n}$, если $\forall\,\mathring{U}_{\varepsilon}(a)$ содержит точки множества $A$.
\par\textbf{Опр.} Множество $A \subset \mathbb{R}^{n}$ называется замкнутым, если оно содержит все свои граничные точки.
\section*{3. Дать определение ограниченного и связного множества в $\mathbb{R}^{n}$.}
\par\textbf{Опр.} Множество $A \subset \mathbb{R}^{n}$ называется ограниченным, если $\exists U_{\varepsilon}((0, 0, \dots, 0))$ точки 0, целиком содержащая множество $A$.
\par\textbf{Опр.} Множество $A \subset \mathbb{R}^{n}$ называется линейно связным, если любые две его точки можно соединить непрерывной кривой.
\section*{4. Дать определение предела ФНП по множеству и непрерывной ФНП.}
\par\textbf{Опр.} Пусть задана функция $f: \mathbb{R}^{n} \to \mathbb{R}^{m}$, множество $A \subset D(f) \subset \mathbb{R}^{n}$ и $a$ - предельная точка множества $A$. Тогда $b \in \mathbb{R}^{m}$ называется пределом функции $f(x)$ в точке $a$ по множеству $A$, если
\begin{itemize}
\item $\forall U_{\varepsilon}(b) \; \exists \mathring{U}_{\delta}(a)$ такая, что $\forall x \in \mathring{U}_{\delta}(a) \cap A \; f(x) \in U_{\varepsilon}(b)$ (определение по Коши)
\item для любой последовательности $\{ a_{k} \}$, $a_{k} \neq a$, сходящейся к точке $a$ и $a_{k} \in A \; \forall k$ последовательность значений $\{ b_{k} \} = \{ f(a_{k}) \}$ сходится к точке $b$ (определение по Гейне)
\end{itemize}
\par\textbf{Опр.} Функция нескольких переменных $f: A \subset \mathbb{R}^{n} \to \mathbb{R}^{m}$ называется непрерывной в точке $a \in A$, предельной для множества $A$, если:$$1) \; \exists \lim_{ \substack{x \to a \\ A}} f(x)$$
$$2) \; \lim_{ \substack{x \to a \\ A}} f(x) = f(a)$$
\par\textbf{Опр.} Функция нескольких переменных $f: A \subset \mathbb{R}^{n} \to \mathbb{R}^{m}$ называется непрерывной на множестве $A$, если она непрерывна во всех точках множества $A$.
\section*{5. Дать определение частной производной ФНП в точке.}
\par\textbf{Опр.} Пусть функция $f: \mathbb{R}^{n} \to \mathbb{R}^{m}$ определена в некоторой $\delta$-окрестности точки $U_{\delta}(a)$ точки $a = (a_{1}, \dots, a_{n}) \in \mathbb{R}^{n}$
\par Пусть $\Delta x_{i}$ - любое приращение $i$-ой переменной функции $f(x_{1}, \dots, x_{n})$ такое, что точка $(a_{1}, \dots, a_{i-1}, a_{i} + \Delta x_{i}, a_{i+1}, \dots, a_{n}) \in U_{\delta}(a)$
\par Частным приращением функции $f$ по переменной $x_{i}$ в точке $a$ называется разность $\Delta_i f(a) = f(a_{1}, \dots, a_{i-1}, a_{i} + \Delta x_{i}, a_{i+1}, \dots, a_{n}) - f(a_{1}, \dots, a_{i-1}, a_{i}, a_{i+1}, \dots, a_{n})$
\par Частной производной функции $f$ по переменной $x_{i}$ в точке $a$ называется предел(если он существует)$$\lim_{ \Delta x_{i} \to 0 } \frac{\Delta_{i}f(a)}{\Delta x_{i}}$$
\par\textbf{Обоз.} $\Large\frac{\partial f(a)}{\partial x_{i}}$ или $f'_{x_{i}}(a)$
\section*{6. Дать определение ФНП, дифференцируемой в точке.}
\par\textbf{Опр.} Функция $f$ называется дифференцируемой в точке $x$, если её полное приращение в некоторой окрестности точки $x$ можно представить в виде:
$$\Delta f(x) = A \cdot \Delta x + \alpha(\Delta x) \cdot |\Delta x|,$$
\par где $A$ - матрица $m \times n$, элементы которой не зависят от $\Delta x$, $\alpha(\Delta x): \mathbb{R}^{n} \to \mathbb{R}^{m}$ - бесконечно малая функция при $\Delta x \to 0$
\section*{7. Записать формулы для вычисления частных производных сложной функции вида $z = f(u(x, y), v(x, y))$.}
$$\large z'_{x} = z'_{u} \cdot u'_{x} + z'_{v} \cdot v'_{x}$$
$$\large z'_{y} = z'_{u} \cdot u'_{y} + z'_{v} \cdot v'_{y}$$
\par или
$$\large \frac{\partial z}{\partial x} = \frac{\partial z}{\partial u} \cdot\frac{\partial u}{\partial x} + \frac{\partial z}{\partial v} \cdot \frac{\partial v}{\partial x}$$
$$\large \frac{\partial z}{\partial y} = \frac{\partial z}{\partial u} \cdot\frac{\partial u}{\partial y} + \frac{\partial z}{\partial v} \cdot \frac{\partial v}{\partial y}$$
\section*{8. Записать формулу для вычисления производной сложной функции вида $u = f(x(t), y(t), z(t))$.}
$$\large \frac{\partial u}{\partial t} = \frac{\partial u}{\partial x} \cdot \frac{\partial x}{\partial t} + \frac{\partial u}{\partial y} \cdot \frac{\partial y}{\partial t} + \frac{\partial u}{\partial z} \cdot \frac{\partial z}{\partial t}$$
\section*{9. Записать формулы для вычисления частных производных неявной функции $z(x, y)$, заданной уравнением $F(x, y, z) = 0$.}
$$\Large \frac{\partial z}{\partial x} = -\frac{\left(\frac{\partial F}{\partial x}\right)}{\left(\frac{\partial F}{\partial z}\right)}$$
$$\Large \frac{\partial z}{\partial y} = -\frac{\left(\frac{\partial F}{\partial y}\right)}{\left(\frac{\partial F}{\partial z}\right)}$$
\section*{10. Сформулировать теорему о связи непрерывности и дифференцируемости ФНП.}
\par\textbf{Теорема.} Если функция $f: \mathbb{R}^{n} \to \mathbb{R}^{m}$ дифференцируема в точке $x \in \mathbb{R}^{n}$, то она непрерывна в точке $x$.
\par\textbf{Следствие.} Если функция $f: \mathbb{R}^{n} \to \mathbb{R}^{m}$ дифференцируема в области $X \in \mathbb{R}^{n}$, то она непрерывна в области $X$.
\section*{11. Сформулировать теорему о необходимых условиях дифференцируемости ФНП.}
\par\textbf{Теорема.} Пусть функция $f: \mathbb{R}^{n} \to \mathbb{R}^{m}$ дифференцируема в точке $x \in \mathbb{R}^{n}$. Тогда в точке $x$ существуют частные производные функции $f$ по всем переменным, то есть определена матрица Якоби $f'(x)$, причём матрица $A$ из опр. дифференцируемой функции и матрица Якоби $f'(x)$ равны, то есть $a_{{ij}} = \frac{\partial f_{j}(x)}{\partial x_{i}}$
\section*{12. Сформулировать теорему о достаточных условиях дифференцируемости ФНП.}
\par\textbf{Теорема.} Пусть функция $f: \mathbb{R}^{n} \to \mathbb{R}^{m}$ имеет матрицу Якоби в некоторой окрестности $U(a)$ точки $a \in \mathbb{R}^{n}$ и все элементы $\frac{\partial f_{j}}{\partial x_{i}}$ матрицы Якоби непрерывны в точке $a \in \mathbb{R}^{n}$. Тогда функция $f$ дифференцируема в точке $a$.
\section*{13. Сформулировать теорему о неявной функции.}
\par\textbf{Теорема.} (формулировка очень большая, здесь очень сильно упрощённый вариант из семинаров)
\par Дано уравнение $F(x, y, z) = 0$. Пусть оно разрешимо относительно $z$, тогда существует неявно заданная функция $z = z(x, y)$, при подстановке которой в уравнение оно обращается в верное равенство, причём дифференцируемая. Её частные производные:
$$\Large \frac{\partial z}{\partial x} = -\frac{\left(\frac{\partial F}{\partial x}\right)}{\left(\frac{\partial F}{\partial z}\right)}$$
\par и
$$\Large \frac{\partial z}{\partial y} = -\frac{\left(\frac{\partial F}{\partial y}\right)}{\left(\frac{\partial F}{\partial z}\right)}$$
\section*{14. Дать определение (полного) первого дифференциала ФНП.}
\par\textbf{Опр.} Пусть функция $f: \mathbb{R}^{n} \to \mathbb{R}^{m}$ определена в некоторой окрестности $U(x)$ точки $x = (x_{1}, \dots, x_{n}) \in \mathbb{R}^{n}$ и дифференцируема в точке $x$. Полным (первым) дифференциалом функции $f$ в точке $x$ называется линейная относительно $\Delta x = (\Delta x_{1}, \dots, \Delta x_{n})$ часть приращения $\Delta f(x)$ функции $f$ в точке $x$.
\par\textbf{Обоз.} $df(x) = f'(x) \cdot \Delta x$
\section*{15. Сформулировать теорему о необходимых и достаточных условиях того, чтобы выражение $P(x, y) dx + Q(x, y) dy$ было полным дифференциалом.}
\par\textbf{Теорема.} Выражение $P(x, y)dx + Q(z, y)dy$ является полным дифференциалом некоторой функции $u(x, y)$ $\iff$
\begin{enumerate}
\item функции $P(x, y), Q(x, y), \large\frac{\partial P(x, y)}{\partial y}, \frac{\partial Q(x, y)}{\partial x}$ непрерывны в некоторой области $G \subset \mathbb{R}^{2}$
\item $\large\frac{\partial P(x, y)}{\partial y} = \frac{\partial Q(x, y)}{\partial x} \; \forall (x, y) \in G$ 
\end{enumerate}
\section*{16. Дать определение второго дифференциала ФНП и матрицы Гессе.}
\par\textbf{Опр.} Пусть функция $f: \mathbb{R}^{n} \to \mathbb{R}^{m}$ определена и дифференцируема в некоторой окрестности $U(x)$ точки $x = (x_{1}, \dots, x_{n}) \in \mathbb{R}^{n}$ и её первый дифференциал $df(x)$ дифференцируем в точке $x$. Вторым дифференциалом функции $f(x)$ в точке $x$ называется дифференциал 1-ого порядка дифференциала функции $f(x)$
\par\textbf{Обоз.} $d^2f(x) = d(df(x))$
\par\textbf{Опр.} Матрицей Гессе функции $f$ называется матрица из частных производных второго порядка этой функции:
$$\left(\begin{matrix}
f''_{x_{1}x_{1}}(x) & \dots & f''_{x_{1}x_{n}}(x) \\
\vdots & & \vdots \\
f''_{x_{n}x_{1}}(x) & \dots & f''_{x_{n}x_{n}}(x)
\end{matrix}\right)$$
\section*{17. Сформулировать теорему о независимости смешанных частных производных от порядка дифференцирования.}
\par\textbf{Теорема.} Пусть скалярная функция $f: \mathbb{R}^{n} \to \mathbb{R}$ имеет в некоторой окрестности $U(a)$ точки $a \in \mathbb{R}^{n}$ смешанные частные производные $f''_{xy}(x)$ и $f''_{yx}(x)$, которые непрерывны в точке $a$. Тогда $f''_{xy}(a) = f''_{yx}(a)$
\section*{18. Дать определение градиента ФНП и производной ФНП по направлению.}
\par\textbf{Опр.} Градиентом функции $f: \mathbb{R}^{n} \to \mathbb{R}$ в точке $x \in \mathbb{R}^n$ называется вектор из частных производных $\text{grad} f(x) = \large \left( \frac{\partial f(x)}{\partial x_{1}}, \dots, \frac{\partial f(x)}{\partial x_{n}} \right)$ , если все частные производные существуют.
\par\textbf{Опр.} Производной функции $f: \mathbb{R}^{n} \to \mathbb{R}$ в точке $a \in \mathbb{R}^n$ по направлению вектора $\vec{n}$ называется число, равное пределу(если он существует):$$\large\frac{\partial f(a)}{\partial \vec{n}} = \lim_{ s \to +0 } \frac{f(a + s\vec{n}) - f(a)}{s}$$
\section*{19. Перечислить основные свойства градиента ФНП.}
\subsection*{Свойства градиента функции и производной по направлению:}
\begin{enumerate}
    \item Если скалярная функция $f: \mathbb{R}^{n} \to \mathbb{R}$ дифференцируема в точке $a \in \mathbb{R}^n$, то $\large \frac{\partial f(a)}{\partial \vec{n}} = \text{пр}_{\vec{n}} \text{grad} f(x)$ - проекция градиента на направление вектора
    \item Если скалярная функция $f: \mathbb{R}^{n} \to \mathbb{R}$ дифференцируема в точке $a \in \mathbb{R}^n$ и $\vec{n} = \text{grad} f$, то $\large \frac{\partial f(a)}{\partial \vec{n}} = |\text{grad} f(a)|$
    \item Если скалярная функция $f: \mathbb{R}^{n} \to \mathbb{R}$ дифференцируема в точке $a \in \mathbb{R}^n$, то в этой точке вектор $\text{grad}f(a)$ указывает направление наибольшего роста функции
    \item Если скалярная функция $f: \mathbb{R}^{n} \to \mathbb{R}$ дифференцируема в точке $a \in \mathbb{R}^n$, то в этой точке вектор $-\text{grad}f(a)$ указывает направление наибольшего убывания функции
    \item Если скалярная функция $f: \mathbb{R}^{n} \to \mathbb{R}$ дифференцируема в точке $a \in \mathbb{R}^n$, то наибольшая скорость возрастания(убывания) функции в точке $a$ равна $|\text{grad}f(a)| \;(-|\text{grad}f(a)|)$
\end{enumerate}
\section*{20. Записать формулу для вычисления производной ФНП по направлению.}
\par Производная функции $f$ по направлению вектора $\vec{n}$ находится как скалярное произведение вектора $\vec{n}$ и градиента функции $\text{grad} f(a)$ в точке $a$ ($\vec{n_{0}}$ - нормированный вектор $\vec{n}$):
$$\large\frac{\partial f(a)}{\partial \vec{n}} = \left( \text{grad}f(a), \vec{n_{0}} \right)$$
\section*{21. Записать уравнения касательной и нормали к поверхности $F(x, y, z) = 0$ в точке $(x_0, y_0, z_0)$.}
\par Касательная к графику функции $F(x, y, z) = 0$ в точке $(x_{0}, y_{0}, z_{0})$:
$$\large F'_{x}(x - x_{0}) + F'_{y}(y - y_{0}) + F'_{z}(z - z_{0}) = 0$$
\par Нормаль к графику функции $F(x, y, z) = 0$ в точке $(x_{0}, y_{0}, z_{0})$:
$$\large\frac{x - x_{0}}{F'_{x}} = \frac{y - y_{0}}{F'_{y}} = \frac{z - z_{0}}{F'_{z}}$$
\section*{22. Сформулировать теорему Тейлора для функции двух переменных.}
\par\textbf{Теорема.} (остаточный член в форме Лагранжа)
\par Пусть скалярная функция $f: \mathbb{R}^{n} \to \mathbb{R}$ имеет в некоторой окрестности $U_{\delta}(x_{0})$ точки $x_{0} \in \mathbb{R}^n$:
\begin{enumerate}
    \item все частные производные до порядка $m + 1$
    \item непрерывные в окрестности $U_{\delta}(x_{0})$
\end{enumerate}
\par Тогда $\forall x \in U_{\delta}(x_{0}) \; \exists \theta \in (0, 1)$:$$\large f(x) = f(x_{0}) + \sum_{k=1}^m \frac{d^k f(x_{0})}{k!} + \frac{d^{m+1}f(x_{0} + \theta(x - x_{0}))}{(m+1)!}$$
\par\textbf{Теорема.} (остаточный член в форме Пеано)
\par Пусть скалярная функция $f: \mathbb{R}^{n} \to \mathbb{R}$ имеет в некоторой окрестности $U_{\delta}(x_{0})$ точки $x_{0} \in \mathbb{R}^n$:
\begin{enumerate}
    \item все частные производные до порядка $m + 1$
    \item причём все частные производные до порядка $m$ непрерывны в окрестности $U_{\delta}(x_{0})$
    \item а все частные производные порядка $m + 1$ непрерывны в точке $x_{0}$
\end{enumerate}
\par Тогда $\forall x \in U_{\delta}(x_{0})$:$$\large f(x) = f(x_{0}) + \sum_{k=1}^m \frac{d^k f(x_{0})}{k!} + o(|x - x_{0}|^m)$$
\section*{23. Дать определение (обычного) экстремума (локального максимума и минимума) ФНП.}
\par\textbf{Опр.} Пусть скалярная функция $f: \mathbb{R}^{n} \to \mathbb{R}$ определена в некоторой окрестности точки $a \in \mathbb{R}^n$. Точка $a$ называется точкой локального максимума (минимума) функции $f(x)$, если $\exists \mathring{U}(a)$ такая, что $\forall x \in \mathring{U}(a) \; f(x) \leq f(a) \; (f(x) \geq f(a))$. Точки локального максимума и локального минимума называются точками локального экстремума функции.
\section*{24. Сформулировать необходимые условия экстремума ФНП.}
\par\textbf{Теорема.} Пусть для скалярной функции $f: \mathbb{R}^{n} \to \mathbb{R}$
\begin{enumerate}
    \item точка $a \in \mathbb{R}^n$ является точкой экстремума
    \item и существует частная производная $f'_{x_{i}}(a)$ для некоторого $i = \overline{1, n}$
\end{enumerate}
\par Тогда $f'_{x_{i}} = 0$
\par\textbf{Следствие 1.} Если $\exists \text{grad} f(a)$, то $\text{grad} f(a) = 0$
\par\textbf{Следствие 2.} Если функция дифференцируема в точке $a$, то $df(a) = 0$
\section*{25. Сформулировать достаточные условия экстремума ФНП.}
\par\textbf{Теорема.} Пусть скалярная функция $f: \mathbb{R}^{n} \to \mathbb{R}$
\begin{enumerate}
    \item дважды непрерывно дифференцируема в некоторой окрестности
    \item $\text{grad} f(a) = 0$
    \item квадратичная форма $d^2f(a)$ 
    \begin{enumerate}
        \item положительно определена, тогда в точке $a$ функция $f(x)$ имеет строгий локальный минимум
        \item отрицательно определена, тогда в точке $a$ функция $f(x)$ имеет строгий локальный максимум
        \item знакопеременна, тогда в точке $a$ функция $f(x)$ не имеет экстремума
    \end{enumerate}
\end{enumerate}
\section*{26. Дать определение условного экстремума ФНП.}
\par\textbf{Опр.} Пусть скалярная функция $f: \mathbb{R}^{n} \to \mathbb{R}$ и векторная функция $\varphi: \mathbb{R}^{n} \to \mathbb{R}^m \; (m < n)$ определены в некоторой окрестности точки $a \in \mathbb{R}^n$
\par $\varphi(x) = \vec{0}$ - некоторое условие
\par Точка $a$ называется точкой условного локального максимума (минимума) функции $f(x)$, если существует проколотая окрестность $\mathring{U}(a)$: $\forall x \in \mathring{U}(a)$, удовлетворяющих условию $\varphi(x) = 0$, $f(x) \leq f(a) \; (f(x) \geq f(a))$.
\par Точки условного максимума и условного минимума называются точками условного локального экстремума.
\section*{27. Сформулировать необходимые условия условного экстремума ФНП.}
\par\textbf{Теорема.} Пусть
\begin{enumerate}
    \item  скалярная функция $f: \mathbb{R}^{n} \to \mathbb{R}$ и векторная функция $\varphi: \mathbb{R}^{n} \to \mathbb{R}^m \; (m < n)$ непрерывно дифференцируемы в некоторой окрестности точки $a \in \mathbb{R}^n$
    \item точка $a$ является точкой условного экстремума функции $f(x)$ при условиях связи $\varphi = 0$
    \item ранг матрицы Якоби $\varphi'(a)$ функции $\varphi(x)$ в точке $a$ равен $m$, то есть $\text{rg } \varphi'(a) = m$
\end{enumerate}
\par Тогда существуют множители Лагранжа $\lambda_{1}, \dots, \lambda_{m}$:
$$\left\{\begin{array}{l}
L'_{x_{1}} (a, \lambda) = 0 \\
\dots \\
L'_{x_{n}} (a, \lambda) = 0 \\
L'_{\lambda_{1}} (a, \lambda) = 0 \\
\dots \\
L'_{\lambda_{m}} (a, \lambda) = 0
\end{array}\right.$$
\par Решения системы являются стационарными точками функции Лагранжа.
\section*{28. Сформулировать достаточные условия условного экстремума ФНП.}
\par\textbf{Теорема.} Пусть
\begin{enumerate}
    \item скалярная функция $f: \mathbb{R}^{n} \to \mathbb{R}$ и векторная функция $\varphi: \mathbb{R}^{n} \to \mathbb{R}^m \; (m < n)$ дважды непрерывно дифференцируемы в некоторой окрестности точки $a \in \mathbb{R}^n$
    \item $\varphi(a) = \vec{0}, \text{rg } \varphi'(a) = m$
    \item координаты точки $(a_{1}, \dots, a_{n}, \lambda_{1}, \dots, \lambda_{m}) \in \mathbb{R}^{(m + n)}$ являются решением системы уравнений:
$$\left\{\begin{array}{l}
L'_{x_{1}} (a, \lambda) = 0 \\
\dots \\
L'_{x_{n}} (a, \lambda) = 0 \\
L'_{\lambda_{1}} (a, \lambda) = 0 \\
\dots \\
L'_{\lambda_{m}} (a, \lambda) = 0
\end{array}\right.$$
\item для функции $L(x) = L(x, \lambda_{a})$ и подпространства $H = \{ dx_{1}, \dots, dx_{n} \; | \; d\varphi(a) = 0 \}$ квадратичная форма $d^2L(a)|_{H}$
    \begin{enumerate}
        \item положительно определённая, тогда функция $f(x)$ в точке $a$ имеет строгий локальный минимум при условии $\varphi(x) = 0$
        \item отрицательно определённая, тогда функция $f(x)$ в точке $a$ имеет строгий локальный максимум при условии $\varphi(x) = 0$
        \item знакопеременная, тогда функция функция $f(x)$ в точке $a$ не имеет условного экстремума при условии $\varphi(x) = 0$
    \end{enumerate}
\end{enumerate}

\section*{29. Дать определение функции Лагранжа и множителей Лагранжа задачи на условный экстремум ФНП.}
\par\textbf{Опр.} Функцией Лагранжа для задачи на условный экстремум
\par $f(x) \to \text{extr}$
\par $\varphi_{1}(x) = 0$
\par $\dots$ 
\par $\varphi_{m}(x) = 0$
\par называется функция $L(x, \lambda) = f(x) + \lambda_{1}\varphi_{1}(x) + \dots + \lambda_{m}\varphi_{m}(x)$.
\par Числа $\lambda_{1}, \dots, \lambda_{n}$ называются множителями Лагранжа.

\end{document}